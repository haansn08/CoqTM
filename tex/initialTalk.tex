%%%%%%%%%%%%%%%%%%%%%%%%%%%%%%%%%%%%%%%%%%%%%%%%%%%%%%%%%%%%
%%  This Beamer template was created by Cameron Bracken.
%%  Anyone can freely use or modify it for any purpose
%%  without attribution.
%%
%%  Modifications by Yannick Forster
%%
%%  Last Modified: September 19, 2014
%%

\documentclass[xcolor=x11names,compress]{beamer}

%% General document %%%%%%%%%%%%%%%%%%%%%%%%%%%%%%%%%%
\usepackage{graphicx}
\usepackage{epstopdf}
\usepackage{tikz}
\usepackage{color}
\usepackage[latin1]{inputenc}
\usepackage[T1]{fontenc}
\usepackage[english]{babel}
\usepackage{etoolbox}
\usepackage[absolute,overlay]{textpos}
\usepackage{everypage}
\usepackage{pgfpages}
\usepackage{xspace}
\usepackage{mathpartir}


%%%%%%%%%%%%%%%%%%%%%%%%%%%%%%%%%%%%%%%%%%%%%%%%%%%%%%

%% Mathematical Notation %%%%%%%%%%%%%%%%%%%%%%%%%%%%%
\usepackage{listings}
\usepackage{lstautogobble}
%\usepackage{amsmath}
\usepackage{amssymb}
\usepackage{tipa}
\usepackage{proof}

\newcommand{\pitem}{\item\pause}

\colorlet{darkgreen}{green!80!black}
\colorlet{shadecolor}{gray!20}
\definecolor{saarblue}{HTML}{2874ae}

\renewcommand{\vec}{\boldsymbol}

%%%%%%%%%%%%%%%%%%%%%%%%%%%%%%%%%%%%%%%%%%%%%%%%%%%%%%

\newtoggle{logo}
\newtoggle{totalcounter}
\newtoggle{section-slides}

\newcommand{\Title}{}
\newcommand{\SubTitle}{}
\newcommand{\Author}{}
\newcommand{\CoAuthor}{}
\newcommand{\Institute}{}

%%%
%%% Shared preamble for all files, e.g. thesis, TikZ standalones, slides, etc.
%%% It defines \macros for types, Turing machines, etc.
%%%

% Packages needed
\usepackage[utf8]{inputenc}
\usepackage{geometry}
\usepackage[small,compact]{titlesec}
\usepackage[final]{listings}
\usepackage{amsmath}
\usepackage[amsmath,hyperref,thmmarks]{ntheorem}
% Warning: The package ntheorem defines a \None macro!
\usepackage{amssymb}
\usepackage{tipa}
\usepackage[english]{babel}
\usepackage{lstautogobble}
\usepackage{proof}
\usepackage{bussproofs}
\usepackage{xparse}
\usepackage{needspace}
\usepackage{xspace}
\usepackage{mathpartir}
\usepackage{stmaryrd} % for |llbracket and \rrbracket
\usepackage{standalone} % useful to out-source graphics


% TikZ ist *kein* Zeichenprogramm.
\usepackage{tikz}
\usetikzlibrary{arrows,shapes,snakes,automata,backgrounds,fit,positioning}
\usepackage{tikz-cd} % for commutative diagrams


%% Formating
\newcommand{\MS}[1]{\ensuremath{\mathsf{#1}}}
\newcommand{\MST}[1]{${\mathsf{#1}}$}
\newcommand{\IsMathMode}{\ifmmode{This is math mode}\else{This is not math mode}\fi}

%% Logic symbols
\newcommand{\defop}{\mathop{:=}}
\newcommand{\imp}{\mathbin{\rightarrow}~}
\newcommand{\Imp}{\mathbin{\Rightarrow}~}
\renewcommand{\iff}{\mathbin{\leftrightarrow}}


% ++ operator:
% Source: https://tex.stackexchange.com/questions/4194/how-to-typeset-haskell-operator-and-friends
\newcommand\doubleplus{+\kern-1.3ex+\kern0.8ex}
\newcommand\mdoubleplus{\ensuremath{\mathbin{+\mkern-10mu+}}}
\newcommand{\app}{\mdoubleplus}

\newcommand{\rew}{\Rightarrow}
\newcommand{\trew}{\stackrel{\textrm{T}}\Rightarrow}
\newcommand{\llrew}{\stackrel{\textrm{L}}\Rightarrow}
\newcommand{\rlrew}{\stackrel{\textrm{R}}\Rightarrow}
\newcommand{\arew}{\triangleright}
\newcommand{\conc}{\mathop{{+}\hskip-5pt{+}}}
\newcommand{\gen}{\Rightarrow}

%% Sets
% \newcommand{\lam}[2]{\lambda#1{.}\hskip.7pt#2}
\newcommand{\setOf}[1]{\bigl\{ #1 \bigr \}}
\newcommand{\setMap}[2]{\setOf{#1~\big|~#2}}
\newcommand{\depPair}[2]{\setOf{#1~{\&}~#2}}
\newcommand{\pair}[2]{\bigl( #1 , #2 \bigr)}
\newcommand{\class}[1]{\bigl[ #1 \bigr]}
\newcommand{\choice}[1]{\bigl< #1 \bigr>}
\newcommand{\explainRel}[2]{\stackrel{\text{#1}}{#2}}
\newcommand{\family}[2]{\bigl( #1 \bigr)_{#2}}
\newcommand{\from}{:}
\renewcommand{\to}{\rightarrow}

%% Types
\newcommand{\Bool}{\mathbb{B}}
\newcommand{\Fin}{\mathbb{F}}
\newcommand{\Nat}{\mathbb{N}}
\newcommand{\Prop}{\mathbb{P}}
\newcommand{\Type}{\mathbb{T}}
\newcommand{\Unit}{\MS{1}}
\newcommand{\Option}{\mathcal{O}}
\newcommand{\List}{\mathcal{L}}
\newcommand{\Rel}{\MS{Rel}}

\newcommand{\True}{\top}
\newcommand{\False}{\bot}

%% Tapes
\newcommand{\tape}[1]{[ #1 ]}
\newcommand{\tapePointer}[1]{\underset{\uparrow}{#1}}
\newcommand{\niltape}{\tape{\tapePointer{}}}
\newcommand{\midtape}[3]{\tape{#1~\tapePointer{#2}~#3}}
\newcommand{\leftof}[2]{\tape{\tapePointer{}~#1~#2}}
\newcommand{\rightof}[2]{\tape{#1~#2~\tapePointer{}}}

% \newcommand{\niltape}{\MS{niltape}}
% \newcommand{\midtape}[3]{\MS{midtape}~#1~#2~#3}
% \newcommand{\leftof}[2]{\MS{leftof}~#1~#2}
% \newcommand{\rightof}[2]{\MS{rightof}~#1~#2}

%% Turing machine types
\newcommand{\Loop}{\MS{loop}}
\newcommand{\Tape}{\MS{Tape}}
\newcommand{\Tapes}[1]{\Tape^{#1}}
\newcommand{\TM}{\MS{TM}}
\newcommand{\Move}{\MS{Move}}
\newcommand{\Act}{\MS{Act}}
\newcommand{\Conf}{\MS{Conf}}
\newcommand{\Tau}{\Gamma}

%% Relations
\newcommand{\rif}{\mathbin{\phi}}
\newcommand{\at} [2][]{#1{|}_{#2}}
\newcommand{\att}[2][]{#1{|\mkern-1.5mu|}_{#2}}
\DeclareMathOperator{\ignoreParam}{\Uparrow}
\DeclareMathOperator{\hideParam}{\Downarrow}


%% Constructors
\DeclareMathOperator{\inl}{\ensuremath{\MS{inl}}}
\DeclareMathOperator{\inr}{\ensuremath{\MS{inr}}}
\newcommand{\Some}[1]{\left\lfloor {#1} \right\rfloor}
% \None is defined sometimes
\renewcommand{\None}{\emptyset}
\newcommand{\true}{\MS{true}}
\newcommand{\false}{\MS{false}}
\newcommand{\unit}{\MS{()}}
\newcommand{\nil}{\MS{nil}}
\newcommand{\cons}{\mathbin{::}}

%% Functions
\newcommand{\map}[2]{\ensuremath{\MS{map}~#1~#2}}
\newcommand{\maptwo}[3]{\ensuremath{\MS{map}_2~#1~#2~#3}}
\newcommand{\rev}[1]{\MS{rev}~#1}

%% Vector
\newcommand{\Vector}[1]{\left[ #1 \right]}
\DeclareMathOperator{\hd}{\ensuremath{\MS{hd}}}
\DeclareMathOperator{\tl}{\ensuremath{\MS{tl}}}
\newcommand{\length}[1]{\left| #1 \right|}
\newcommand{\blength}[1]{\bigl| #1 \bigr|}


%%
%% Encding
%%
\newcommand{\contains}{\simeq}
\newcommand{\size}[1]{\length{encode(#1)}}

%% Semantics
\newcommand{\terminates}{\mathrel{\triangleright}}
\newcommand{\TerminatesIn}{\mathrel{\downarrow}}
\newcommand{\Realise}{\mathrel{\vDash}}
\newcommand{\RealiseIn}[1]{\mathrel{\vDash^{#1}}}

%%
%% Turing Machines
%%

%% Control flow operators
\newcommand{\While}{\MS{While}}
\newcommand{\Seq}{;~}
\newcommand{\Match}{\MS{Match}}
\newcommand{\If}[3]{\MS{If}~#1~\MS{Then}~#2~\MS{Else}~#3}
\newcommand{\Let}[2]{\MS{let}~#1~\MS{in}~#2}
\newcommand{\cond}[3]{\MS{if}~#1~\MS{then}~#2~\MS{else}~#3}
\newcommand{\Nop}{\MS{Nop}}
\newcommand{\Return}[2]{\MS{Return}~#1~#2}
% \newcommand{\Return}[2]{\MS{Return}_{#2}~#1}

%% Lifts
% #1 is the machine, #2 the lifting
\newcommand{\LiftTapes}[2]{\mathop{\Uparrow_{#2}} #1}
\newcommand{\LiftAlphabet}[2]{\mathop{\Uparrow_{#2}} #1}
% #1 is the machine, #2 the alphabet lifting, and #3 the tape-lifting
\newcommand{\LiftBoth}[3]{\mathop{\Uparrow_{#2;~#3}} #1}




%%%
%%% lstlisting
%%%

% Style and language to define complex multi-line definitions similar to Coq code
\lstdefinelanguage{semicoq}{
  keywords={if,then,else,true,false,match,Match,If,Then,Else,Nop,Return,Move,Reset,DoAct,WriteMove,L,R,N},
  comment=[s]{(*}{*)},
}

%% Overlap #2 over phantom #1, e.g.
%% % XX\phalign{abcdefg}{YY}XX \\
%% % XXabcdefgXX
%% gets
%% XXYY     XX
%% XXabcdefgXX
%% Idea from https://tex.stackexchange.com/questions/212710/fill-space-created-by-phantom-with-other-text
\newcommand{\phalign}[2]{\makebox[0pt][l]{\ensuremath{#2}}\phantom{#1}}

\lstdefinestyle{semicoqstyle}{
  mathescape=true,
  keywordstyle=\textsf,
  language=semicoq,
  literate={
    {=>}{{$\Rightarrow$}}2
    {>->}{{$\rightarrowtail\,$}}2
    {<->}{{$\leftrightarrow$ }}2
    {->}{{$\to$ }}3
    {~}{{$\lnot$}}1
    {/\\}{{$\land$}}2
    {\\/}{{$\lor$}}2
    {forall}{{$\forall$}}1
    {exists}{{$\exists$}}1
    {<>}{{$\not =$}}{1}
    {<=}{{$\leq$}}{1}
    {<}{{$\lt$}}{1}
    {>=}{{$\ge$}}{1}
    {>}{{$\gt$}}{1}
    {[}{{$[$}}{1}
    {|}{{$|$}}{1}
    {]}{{$]$}}{1}
    {])}{{$])$}}{2}
    {(}{{$($}}{1}
    {)}{{$)$}}{1}
    {match}{{$\MS{match}$}}5
    {if}{{$\MS{if}$}}1
    {then}{{$\phalign{\MS{else}}{{\MS{then}}}$}}3
    {else}{{$\phalign{\MS{else}}{{\MS{else}}}$}}3
    {If}{{$\MS{If}$}}2
    {Then}{{$\phalign{\MS{Else}}{{\MS{Then}}}$}}4
    {Else}{{$\phalign{\MS{Else}}{{\MS{Else}}}$}}4
  }
}

\lstdefinelanguage{pseudocode}{
  keywords={If,Then,Else,Do,While,Reset,Return,Continue,Break},
}

\lstdefinestyle{pseudocode}{
  mathescape=true,
  language=pseudocode,
  literate={
    {:=}{{$\leftarrow$}}{2}
    {<>}{{$\not =$}}{1}
    {<=}{{$\leq$}}{1}
    {<}{{$\lt$}}{1}
    {>=}{{$\ge$}}{1}
    {>}{{$\gt$}}{1}
  }
}






%%% Local Variables:
%%% mode: LaTeX
%%% TeX-master: "thesis"
%%% End:
%% My favourite color theme for beamer
\usecolortheme{orchid}
% \usepackage{verbatim}

%% Use "mathematical" versions of the type names
\renewcommand{\Type}{\mathbb{T}}
\renewcommand{\Nat}{\mathbb{N}}
\renewcommand{\Bool}{\mathbb{B}}
\renewcommand{\Option}{\mathcal{O}}
\renewcommand{\List}{\mathcal{L}}
\newcommand{\Pow}{\mathcal{P}}

\beamertemplatenavigationsymbolsempty%
% \setbeamertemplate{footline}[page number]
%% Show only current page number
\setbeamertemplate{footline}{\raisebox{5pt}{\makebox[\paperwidth]{\hfill\makebox[20pt]{\scriptsize{\color{gray}\insertframenumber}}}}}

%% Display a list of references at the bottom
\newcommand\refs[1]{%
  \begin{textblock*}{8cm}(0.3cm,9.0cm)%
    \scriptsize {\color{gray}#1}
  \end{textblock*}
}


\title{Formalising multi-tape Turing machines in Coq}
\subtitle{First Bachelor Seminar Talk}
\author{Maximilian Wuttke}
\institute{Saarland University\\
  \bigskip
  \tiny
  Programming Systems Lab
}
\date{December 21, 2017\\
  \bigskip{}
  \bigskip{}
  {
    \tiny
    Advisor: Yannick Forster\\
    Supervisor: Prof.\ Dr.\ Gert Smolka%
  }
}


% Passt nicht gut
% \include{beamerlayout}

\begin{document}

\frame{\titlepage}

\begin{frame}{Motivation: Finish the reduction chain}
  \[\MS{L} \rightarrow m\MS{TM} \rightarrow \MS{TM} \rightarrow \MS{PCP}\]

  \begin{itemize}
    \pause\item Reduction of $\MS{TM} \rightarrow \MS{PCP}$ formalised in Coq%
    \footnote{\tiny Yannick Forster, Edith Heiter, Gert Smolka: Verification of PCP-Related Computational Reductions in Coq; 2017}
    \pause\item Computational theory of $L$ formalised in Coq%
    \footnote{\tiny Yannick Forster and Gert Smolka: Weak Call-by-Value Lambda Calculus as a Model of Computation in Coq; 2017}
    \pause\item Now we formalise multi-tape Turing machines%
    \pause\item Goal: build a multi-tape Turing machine that simulates $L$%
    \footnote{\tiny Yannick Forster, Fabian Kunze, Marc Roth: The strong invariance thesis for a lambda-calculus; LOLA 2017}%
\end{itemize}
\end{frame}

\begin{frame}{Introduction: Turing machines}
  Pros:
  \begin{itemize}
    \item are easy to understand / imagine;
    \item are the de-facto standard model of computation
  \end{itemize}
  \pause%
  Cons:
  \begin{itemize}
    \item not compositional;
    \item encoding data is tedious;
    \item formal reasoning is tedious;
    \item completely unstructured
  \end{itemize}
\end{frame}

\begin{frame}{Introduction: Related Work}
  \footnotesize
  \begin{thebibliography}{10}
    \beamertemplatearticlebibitems%
    \bibitem{}
    Xu, Jian and Zhang, Xingyuan and Urban, Christian
    \newblock{\em Mechanising Turing Machines and Computability Theory in Isabelle/HOL}%
    \newblock{ITP 2013}

    \bibitem{}
    Andrea Asperti and Wilmer Ricciotti
    \newblock{\em A formalization of multi-tape Turing machines}%
    \newblock{Theoretical Computer Science, 2015}

    \bibitem{}
    Alberto Ciaffaglione
    \newblock{\em Towards Turing computability via coinduction}%
    \newblock{Science of Computer Programming, 2016}
  \end{thebibliography}
\end{frame}


\begin{frame}{Definitions}
  Let $\Sigma$ be a finite type and $n:\Nat$.
  \begin{definition}[$n$-tape Turing machine over $\Sigma$]
    An $n$-tape Turing machine over $\Sigma$ is a record $\MS{mTM}~\Sigma~n$:
    \begin{itemize}
      \item $Q$ is the finite type of states;
      \item $\gamma \from Q \times \Option(\Sigma)^n \to Q \times (\Option(\Sigma) \times \MS{Move})^n$
        \\ (transition function);
      \item $s:Q$ (start state);
      \item $h \from Q \to \Bool$ (halting states)
    \end{itemize}
    Where $\MS{Move} := \setOf{L, R, N}$.
  \end{definition}
  \refs{[Asperti and Ricciotti]} % use this to show references on a slide
\end{frame}

\begin{frame}{Definitions}
  \begin{definition}[Tape]
    % XXX List
    A tape over $\Sigma$ is either:
    $ \niltape$, $\leftof{r}{R}$, $\midtape{R}{m}{L}$, or $\rightof{L}{l}$,\\
    Where $r,l : \Sigma$ and $R,L : \List(\Sigma)$.
    \pause%
    \begin{alignat*}{2}
      \MS{Tape} := ~ & \MS{niltape} \\
      | ~ & \MS{leftof}  ~ (r:\Sigma) ~ (R:\List(\Sigma)) \\
      | ~ & \MS{midtape} ~ (L:\List(\Sigma)) ~ (m:\Sigma) ~ (R:\List(\Sigma)) \\
      | ~ & \MS{rightof} ~ (l:\Sigma) ~ (L:\List(\Sigma)).
    \end{alignat*}
  \end{definition}
  \pause%
  \begin{definition}[Configuration]
    A configuration is a record $\MS{Conf}_M := \{\MS{state}:Q_M;~ \MS{tapes}:\Tape^n\}$.
  \end{definition}
  \refs{[Asperti and Ricciotti]} % use this to show references on a slide
\end{frame}

\begin{frame}{Definitions}
  \begin{definition}[Step function]
    With the functions
    \begin{alignat*}{2}
      % \MS{mv}      & \from \Move \to \Tape \to \Tape \\
      % \MS{wr}      & \from \Tape \to \Option~\Sigma \to \Tape \\
      \MS{wr\_mv}  & \from \Tape \to (\Option(\Sigma) \times \Move) \to \Tape \\
      \MS{current} & \from \Tape \to \Option(\Sigma)
    \end{alignat*}
    we define the \textbf{step function} $\MS{step}_M \from \MS{Conf}_M \to \MS{Conf}_M$:
    \begin{alignat*}{2}
      \MS{step}_M &~(\MS{tapes}, q) &~:=~& \mlet{(q', \MS{actions}) := \gamma_M(q, \map~\MS{current}~\MS{tapes})}{ \\
                  &                 &~  ~& (q', \map_2~\MS{wr\_mv}~\MS{tapes}~\MS{actions})}
      \end{alignat*}
  \end{definition}
  \refs{[Asperti and Ricciotti]} % use this to show references on a slide
\end{frame}

\begin{frame}{Definitions}
  \begin{definition}[Execution]
    \begin{align*}
      \MS{loop}~k~f~h~s :=
      \begin{cases}
        \Some{s}                  & h(s) = \true \\
        \None                     & h(s) = \false \land k = 0 \\
        \MS{loop}~(k-1)~f~h~(f(s)) & h(s) = \false \land k > 0
      \end{cases}
    \end{align*}
    \pause%
    \begin{alignat*}{4}
      \MS{iconf}_M ~ t_{in}          &:= (\MS{tapes}, s_M) \\
      \MS{hconf}_M ~ (\MS{tapes}, q) &:= h_M(q) \\
      \MS{exec}_M  ~ t_{in}~k        &:= \MS{loop}~k~\MS{step}_M~\MS{hconf}~(\MS{initc}_M~t_{in})
    \end{alignat*}
    {
      \small
      $\MS{exec}_M \from \Tape^n \to \Nat \to \Option(\MS{Conf})$
    }
  \end{definition}

  \pause%
  We write $\mathbf{M(t_{in}) \triangleright^n c_{out}}$, if $\MS{exec}_M~t_{in}~n= \Some{c_{out}}$.
  \refs{[Asperti and Ricciotti]} % use this to show references on a slide
\end{frame}

\begin{frame}{Correctness of machines}
  \begin{definition}[Realisation, $M \VDash_f R$]
    Let $R \subseteq (\Tape^n) \times (F \times \Tape^n)$ and $f \from Q_M \to F$, for a finite $F$.\\
    \begin{multline*}
      \mathbf{M \VDash_f R} :=
      \forall (t_{in} : \Tape^n)~(i:\Nat)~(q_{out}:Q_M)~(t_{out} : \Tape^n). \\
      M(t_{in}) \triangleright^i (q_{out}, t_{out}) \rightarrow
      \left(t_{in}, \left(f(q_{out}), t_{out} \right) \right) \in R.
    \end{multline*}
    Then we say that $M$ \textbf{realises} $R$.
  \end{definition}
  \pause%
  For \textbf{partioned machines} $M : \setOf{M' : \MS{mTM}~\Sigma~n;~ f \from Q_{M'} \to F}$,\\
  we write $\mathbf{M \VDash R}$.
\end{frame}

\begin{frame}{Termination of machines}
  \begin{definition}[Termination, $M \downarrow T$]
    Let $T \subseteq (\Tape^n) \times \Nat$.
    \[
      \mathbf{M \downarrow T} :=
      \forall (t_{in},~i) \in T.~
      \exists c_{out},~M(t_{in}) \triangleright^i c_{out}
    \]
    We say that $M$ \textbf{terminates in} $T$.
  \end{definition}
\end{frame}

\begin{frame}{Relations}
  \footnotesize
  Let $R, S \subseteq X \times Y$; $T \subseteq Y \times Z$; $U \subseteq X \times X$; $V \subseteq X \times (F \times Y)$; and
  $\Phi \from F \to \Pow(X \times Y)$.
  \vspace{0.5cm}

  \begin{columns}
    \begin{column}{0.6\textwidth}
      Standard relation combinators:
      \begin{alignat*}{2}
        R \cap  S &:= \setMap{(x, y)}{(x, y) \in R \land (x,y) \in S} \\
        R \cup  S &:= \setMap{(x, y)}{(x, y) \in R \lor  (x,y) \in S} \\
        R \circ T &:= \setMap{(x, z)}{\exists y,~ (x,y) \in R \land (y,z) \in T} \\
        \MS{Id}_X &:= \setMap{(x, x)}{x : X}
      \end{alignat*}
      Reflexive transitive closure:
      \[\inferrule{ }{(x,x) \in U^*} \qquad \inferrule{(x, y) \in U \and (y,z) \in U^*}{(x, z) \in U^*}\]
      Big union:
      \[\bigcup_{a:F}\Phi(a) := \setMap{(x, y)}{\exists a, (x,y) \in (\Phi~a)}\]
    \end{column}
    \begin{column}{0.4\textwidth}
      Parametrise relations:
      \begin{alignat*}{2}
        \ignoreParam R &:= \setMap{(x, (a, y))}{a : F \land (x, y) \in R} \\
        \hideParam   R &:= \setMap{((x, a), y)}{a : F \land (x, y) \in R}
      \end{alignat*}
      Relational if:
      \begin{alignat*}{3}
        R \rif S :=  &\setMap{(x, (\true,  y))&&}{(x, y) \in R} \\
              \cup\, &\setMap{(x, (\false, y))&&}{(x, y) \in S}
      \end{alignat*}
      Restriction: {\tiny(for a fixed $a : F$)}
      \begin{alignat*}{3}
        V\at {a} &:= \setMap{(x,      z)}{(x,(a, z)) \in V} \\
        % V\att{a} &:= \setMap{(x, (a, z))}{(x,(a, z)) \in V} \\
      \end{alignat*}
    \end{column}
  \end{columns}
\end{frame}

\begin{frame}{Combinators}
  \begin{itemize}
    \item Implement programming primitives as operators on Turing machines:
    \begin{itemize}
      \item match
      \item sequential composition
      \item if then else
      \item do-while
    \end{itemize}
  \item Verify them using correctness relations
  \end{itemize}
  \refs{[Asperti and Ricciotti]} % use this to show references on a slide
\end{frame}

\begin{frame}{Sequential composition and If Then Else}
  Let $M_1$ be parametrised over $F$ and $M_2$ over $F'$.
  \begin{corollary}[Correctness of sequential composition]
    If $M_1 \VDash R_1$ and $M_2 \VDash R_2$, then $(M_1 \mseq M_2) \VDash R_1 \circ \hideParam R_2$.
  \end{corollary}
  \bigskip
  \pause
  Let $M_1$ be parametrised over $\Bool$ and $M_2$ and $M_3$ over $F$.
  \begin{corollary}[Correctness of boolean if]
    If $M_1 \VDash R_1$ and $M_2 \VDash R_2$ and $M_3 \VDash R_3$, then
    $(\mif{M_1}{M_2}{M_3}) \VDash (R_1 \at \true) \circ R_2 \cup (R_1 \at \false) \circ R_3$.
  \end{corollary}
\end{frame}

\begin{frame}{Machine transformations}
  \textbf{Problem:}
  When combining machines, the numbers of tapes and the alphabet have to match!
  \pause%
  \bigskip

  \textbf{Solution:} Two operations on machines:
  \begin{itemize}
    \item $n$-Lift: add/rearange tapes
    \item $\Sigma$-Lift: translate symbols
  \end{itemize}
\end{frame}

\begin{frame}{$n$-Lift}
  Let $f \from \Fin_m \hookrightarrow \Fin_n$ be an injection between tape indexes.

  Let $M$ be an $m$-tape $F$-partitioned Turing machine parametrised over $\Sigma$,
  and $R$ be a $F$-parametrised relation.

  Define an $n$-tape Turing machine $\MS{Lift}_f~M$ and a relation $\MS{Lift}_f~R$.
  \begin{align*}
    \MS{Lift}_f~R := \setMap{(t_{in}, (a, t_{out})) &}{ \bigl(f^{-1}(t_{in}), (a, f^{-1}(t_{out})) \bigr) \in R } \cap \\
    \Uparrow \setMap{(t_{in}, t_{out}) &}{ \forall i \notin \MS{img}~f.~ t_{in}[i] = t_{out}[i] }
  \end{align*}
  {\footnotesize (Where $f^{-1} \from \Tape^m \to \Tapes^n$)}
  % (Tapes that are not in the image of $f$ do nothing, the other tapes simulate the corresponding tapes of $M$.)

  \begin{lemma}[Correctness of the $n$-Lift]
    \[
      M \VDash R \rightarrow (\MS{Lift}_f~M) \VDash (\MS{Lift}_f~R)
    \]
  \end{lemma}

\end{frame}

\begin{frame}{$\Sigma$-Lift}
  Let $f \from \Sigma \hookrightarrow \Tau$ be an injection between alphabets.

  Let $\mathbf{default} : \Sigma$.

  Let $M$ be an $m$-tape $F$-partitioned Turing machine parametrised over $\Sigma$,
  and $R$ be a $F$-parametrised relation.

  Define an $n$-tape Turing machine $\MS{Lift}_{f,\MS{default}}~M$ and a relation $\MS{Lift}_{f,\MS{default}}~R$.
  \begin{align*}
    \MS{lift}_{f,\MS{default}}~R := \setMap{(t_{in}, (a, t_{out}))}{ (f^{-1}_{\MS{default}} (t_{in}) , (a, f^{-1}_{\MS{default}} (t_{out}))) \in R}
  \end{align*}

  \begin{lemma}[Correctness of the $\Sigma$-Lift]
    \[
      M \VDash R \rightarrow (\MS{Lift}_{f, \MS{default}}~M) \VDash (\MS{Lift}_{f, \MS{default}}~R)
    \]
  \end{lemma}
\end{frame}

\begin{frame}{Encoding}
  \begin{definition}[Encodable types]
    A type $X$ is \emph{encodable} over $\Sigma$, if there exists a function $\MS{encode} \from X \to \List~\Sigma$, s.t.
    \begin{multline*}
      \forall (v_1~v_2:X)~(r_1~r_2: \List(\Sigma)). \\
      \MS{encode}(x) \app r_1 = \MS{encode}(v_2) \app r_2 \rightarrow
      v_1 = v_2 \land r_1 = r_2
    \end{multline*}
  \end{definition}
  \pause%
  \begin{definition}[Tape encodes value]
    We write $\Sigma^+$ instead of $\Sigma + \Bool$. \\
    For a $X$ encodable over $\Sigma$, a $\Sigma^+$-tape $t$ \textbf{encodes} a value $x:X$, if
    \[t = \midtape{(r_1 \app [\inr \true])}{(\inl y)}{(\map~\inl~ys \app \inr \false \cons r_2)}\]
    for $y \cons ys = \MS{encode}(x)$, and some $r_1, r_2 \in \List~\Sigma^+$.
  \end{definition}
\end{frame}

\begin{frame}{Calculate functions}
  \begin{definition}[Calculate function]
    A machine $M$ \emph{computes} a function $f \from X \to Y$ from $i < n$ to $j < n$, if $M \VDash \MS{FunRel}_{f,i,j}$ with
    %\small
    \begin{multline*}
      \MS{FunRel}_{f,i,j} := \\
      \ignoreParam \setMap{(t_{in}, t_{out})}{
      \forall x:X,~ \MS{encodes}(t_{in}[i], x) \rightarrow \MS{encodes}(t_{out}[j], f(x))}
    \end{multline*}
  \end{definition}
  % \pause%
  % \begin{lemma}[Finite functions]
  %   All functions with finite range and domain are computable in one step.
  % \end{lemma}
  \pause%
  Extendable to multiple tapes.
\end{frame}

\begin{frame}{Example: De Morgan machine}
  Assume $\MS{AND} : \MS{mTM}~\Bool^+~2$ that computes $andb \from \Bool \to \Bool \to \Bool$ from $0$ and $1$ to $1$.

  Assume $\MS{NOT} : \MS{mTM}~\Bool^+~1$ that computes $notb \from \Bool \to \Bool$.
  \[
    \MS{OR} := \MS{Lift}_{[0 \mapsto 0]}~\MS{NOT} \mseq \MS{Lift}_{[0 \mapsto 1]} \MS{NOT} \mseq \MS{AND} \mseq \MS{Lift}_{[0 \mapsto 1]}~\MS{NOT}
  \]
  \pause%
  \textbf{Fact:} \MS{OR} computes $\orb \from \Bool \to \Bool \to \Bool$ from $0$ and $1$ to $1$.

  \textbf{Proof:}
  Using the correctness lemmas about $\MS{Lift}$ and sequential composition we get a relation $R'$ that $M$ realises.
  Show $R' \subseteq \MS{FunRel}_{orb,i,j}$.
  This follows with the De Morgan law.
\end{frame}

\begin{frame}{Example: De Morgan machine 2}
  \textbf{Fact:}
  \[
    OR' := \MS{Lift}_{[\true \mapsto \false, \false \mapsto \true]}~\MS{AND}
  \]
  also computes $orb$ from $0$ and $1$ to $1$.

  \pause%
  \begin{table}
    \begin{tabular}{|cc|cc|c|c|c|}
      \hline
      $a$ & $\bar{a}$ & $b$ & $\bar{b}$ & $a\land b$ & $\bar{a}\land\bar{b}$ & $\overline{\bar{a}\land\bar{b}}=a\lor b$\tabularnewline
      \hline
      0 & 1 & 0 & 1 & 0 & 1 & 0\tabularnewline
      0 & 1 & 1 & 0 & 0 & 0 & 1\tabularnewline
      1 & 0 & 0 & 1 & 0 & 0 & 1\tabularnewline
      1 & 0 & 1 & 0 & 1 & 0 & 1\tabularnewline
      \hline
    \end{tabular}
  \end{table}
\end{frame}

\begin{frame}{Other results}
  \begin{itemize}
    \item Termination relations
    \item $\MS{match}$
    \item $\MS{while}$
    \item $\MS{mirror}$
    \item Basic machines (move, read, write, nop)
    \item \textbf{Fact:} All finite functions can be computed in 1 step.
  \end{itemize}
\end{frame}

\begin{frame}{Conclusion}
  What we have
  \begin{itemize}
    \item Formal definition of Turing machines and a framework of correctness predicates;
    \item Construction and verification of combinators and transformations;
    \item Formal definition of encoding and computation;
    \item Implementation of some basic and compound machines
  \end{itemize}

  Future work
  \begin{itemize}
    \item Verify compound machines with $\MS{while}$, e.g.\ the projection functions for tuples and lists;
    \item Build towards an interpreter for the weak CBV lambda calculus $L$ with time and space analysis;
    \item Build a Universal Turing machine
  \end{itemize}
\end{frame}

\appendix % appendix sections will not occur in the headline

%% Backup-Slides

\begin{frame}{Backup Slides}
  This Are The Backup Slides!
  \vfill
  Thank you!
\end{frame}

\begin{frame}{Related Work}
  \footnotesize
  \begin{thebibliography}{10}
    \beamertemplatearticlebibitems%
    \bibitem{}
    Xu, Jian and Zhang, Xingyuan and Urban, Christian
    \newblock{\em Mechanising Turing Machines and Computability Theory in Isabelle/HOL}%
    \newblock{ITP 2013}
    \bibitem{}
    Andrea Asperti and Wilmer Ricciotti
    \newblock{\em A formalization of multi-tape Turing machines}%
    \newblock{Theoretical Computer Science, 2015}
    \bibitem{}
    Alberto Ciaffaglione
    \newblock{\em Towards Turing computability via coinduction}%
    \newblock{Science of Computer Programming, 2016}
    \bibitem{}
    Yannick Forster and Gert Smolka
    \newblock{\em Weak Call-by-Value Lambda Calculus as a Model of Computation in Coq}%
    \newblock{ITP 2017}
    \bibitem{}
    Yannick Forster, Fabian Kunze, Marc Roth
    \newblock{\em The strong invariance thesis for a lambda-calculus}%
    \newblock{LOLA 2017}
    \bibitem{}
    Yannick Forster, Edith Heiter, Gert Smolka
    \newblock{\em Verification of PCP-Related Computational Reductions in Coq}%
    \newblock{(Pre-print)}
  \end{thebibliography}
\end{frame}

\begin{frame}{Match}
  \framesubtitle{The idea}
  Given:
  \begin{itemize}
    \item Two finite types $F$ and $F'$,
    \item A machine $M$ and a partitioning function $p \from Q_M \to F$,
    \item For all $a:F$, a machine $M'_a$ with a partitioning function $p'ay \from Q_{M'_a} \to F'$
  \end{itemize}
  Do:
  \begin{enumerate}
    \item Execute a machine $M$.
    \item Depending on the state of termination $q_h$:
      \begin{itemize}
        \item Execute the machine $M'_{p(q_h)}$.
      \end{itemize}
  \end{enumerate}
\end{frame}

\begin{frame}{Match}
  \framesubtitle{Example for a match machine}
	\vspace{-1cm}
  \begin{tikzpicture}[->,>=stealth',shorten >=1pt,auto,node distance=2.8cm]
  \begin{scope}
	  % Machine M
	  \node[state]          (M init)                                    {$s$};
	  \node[state]          (M exit 1)  [right of=M init,yshift= 1.5cm] {$h_1$};
	  \node[state]          (M exit 2)  [right of=M init,yshift= 0.0cm] {$h_2$};
	  \node[state]          (M exit 3)  [right of=M init,yshift=-1.5cm] {$h_3$};
	  \path (M init)
	  edge[dotted] (M exit 1)
	  edge[dotted] (M exit 2)
	  edge[dotted] (M exit 3);
	  \path (M init) ++(-1.0,0) edge (M init);
  \end{scope}
  \begin{scope}[xshift=6.5cm]
	  % Match-Machines
	  \begin{scope}[yshift=1.5cm]
		  % Accepting match machine
		  \node[state]          (M 1 init)                                        {$s'$};
		  \node[state, double]  (M 1 exit 1)  [right of=M 1 init, yshift= 0.75cm] {$h'_1$};
		  \node[state, double]  (M 1 exit 2)  [right of=M 1 init, yshift=-0.75cm] {$h'_2$};
		  \path (M 1 init)
		  edge[dotted] (M 1 exit 1)
		  edge[dotted] (M 1 exit 2);
	  \end{scope}
	  \begin{scope}[yshift=-1.5cm]
		  % Accepting match machine
		  \node[state]          (M 2 init)                                        {$s''$};
		  \node[state, double]  (M 2 exit 1)  [right of=M 2 init, yshift= 0.75cm] {$h''_1$};
		  \node[state, double]  (M 2 exit 2)  [right of=M 2 init, yshift=-0.75cm] {$h''_2$};
		  \path (M 2 init)
		  edge[dotted] (M 2 exit 1)
		  edge[dotted] (M 2 exit 2);
	  \end{scope}
  \end{scope}
  % Connecting edges
  \path
  (M exit 1) edge node[anchor=south] {$(\None, N)$} (M 1 init)
  (M exit 2) edge node[anchor=north,yshift=-0.2cm] {$(\None, N)$} (M 1 init)
  (M exit 3) edge node[anchor=north] {$(\None, N)$} (M 2 init);

  \begin{pgfonlayer}{background}
	  \filldraw [line width=4mm,join=round,blue!10]
	  (M   exit 1.north -| M   init.west) rectangle (M   exit 3.south -| M   exit 3.east)
	  (M 1 exit 1.north -| M 1 init.west) rectangle (M 1 exit 2.south -| M 1 exit 2.east)
	  (M 2 exit 1.north -| M 2 init.west) rectangle (M 2 exit 2.south -| M 2 exit 2.east);
  \end{pgfonlayer}
\end{tikzpicture}

  \[
    p(h_1) = \true \qquad p(h_2) = \true \qquad p(h_3) = \false
  \]
\end{frame}

\begin{frame}{Match}
  \framesubtitle{Definition}
  \footnotesize
  $$Q_{\MS{match}} := Q_M + \depPair{y:F}{Q_{ \left( M'_{y} \right) }}$$
  \begin{alignat*}{2}
    \gamma_{\MS{match}} & (\inl q, \MS{symbols}) &&:=
    \begin{cases}
      \gamma_{M}(q, \MS{symbols})                                & h_M(q) = \false \\
      \left( s_{\left(M'_{p(q)}\right)}, (\None, N)^n \right)  & h_M(q) = \true
    \end{cases} \\
    \gamma_{\MS{match}} & (\inr (a, q), \MS{symbols}) &&:= \gamma_{\left({M'_a}\right)}(q, \MS{symbols})
  \end{alignat*}
  \begin{alignat*}{3}
    h_{\MS{match}} &~ (\inl      q) &&:= \false \\
    h_{\MS{match}} &~ (\inr (a, q)) &&:= h_{\left( M'_a \right)}(q)
  \end{alignat*}
  \begin{alignat*}{3}
    p_{\MS{match}} &~ (\inl      q) &&:= \dots \\
    p_{\MS{match}} &~ (\inr (a, q)) &&:= p'_a(q)
  \end{alignat*}
\end{frame}

\begin{frame}{Match}
  \framesubtitle{Correctness}
  \begin{lemma}[Correctness of $\MS{match}$]
    If $M \VDash R$ and for each $a:F$, $M'_a \VDash R'_a$, then
    $$\MS{match} \VDash \bigcup_{y:F} \left( (R \at y) \circ R'_a \right)$$
  \end{lemma}
\end{frame}


\begin{frame}{Sequential composition and If Then Else}

  \begin{definition}[Sequential composition]
    $$M_1 \mseq M_2 := \mmatch~M_1~(\lam y M_2)$$
  \end{definition}

  \begin{definition}[Boolean if]
    $$ \mif{M_1}{M_2}{M_3} := \MS{match}~(M_1) \left(\lam b \begin{cases} M_2 & b = \true \\ M_3 & b = \false \end{cases} \right) $$
  \end{definition}

\end{frame}

\begin{frame}{While}
  \begin{columns}

    \begin{column}{0.6\textwidth}
      Idea:
      \begin{itemize}
        \item Let $M$ be parametrised over $\Bool \times F$.
        \item Repeat $M$ until terminates in a rejecting state
        \item Partition function returns the $F$ part
      \end{itemize}
    \end{column}
    \pause%
    \begin{column}{0.4\textwidth}
      Example:
      \begin{figure}
        \resizebox{3.5cm}{!}{%
          \begin{tikzpicture}[->,>=stealth',shorten >=1pt,auto,node distance=2.8cm,bend angle=45]
            \begin{scope}
              % Machine M
              \node[state]          (M init)                                    {$q_0$};
              \node[state]          (M exit 1)  [right of=M init,yshift= 1.5cm] {$q_{h1}$};
              \node[state, double]  (M exit 2)  [right of=M init,yshift= 0.0cm] {$q_{h2}$};
              \node[state, double]  (M exit 3)  [right of=M init,yshift=-1.5cm] {$q_{h3}$};
              \path (M init)
              edge[dotted]          (M exit 1)
              edge[dotted]  (M exit 2)
              edge[dotted]  (M exit 3);
              \path (M init) ++(-1.0,0) edge (M init);
              \path (M exit 1) edge[bend right] node[anchor=south,yshift=0.2em] {$(\None, N)$} (M init);
            \end{scope}

            \begin{pgfonlayer}{background}
              \filldraw [line width=4mm,join=round,blue!10]
              (M   exit 1.north -| M   init.west) rectangle (M   exit 3.south -| M   exit 3.east);
            \end{pgfonlayer}
          \end{tikzpicture}
        }
      \end{figure}
      \footnotesize
      \vspace{-0.2cm}
      \begin{align*}
        \#1(p(q_{h1})) &= \true \\
        \#1(p(q_{h2})) &= \#1(p(q_{h3})) = \false
      \end{align*}
    \end{column}
  \end{columns}
  \pause%
  \begin{lemma}[Correctness of $\MS{while}$]
    If $M \VDash R$, then
    \[
      \MS{while} \VDash
      \left( \bigcup_{y:F} R\at{(\true, y)} \right)^* \circ
      \setMap{(t, (y, t'))}{(t, ((\false, y), t')) \in R}
    \]
  \end{lemma}
\end{frame}


\begin{frame}{Encoding functions}
  \footnotesize
  \begin{alignat*}{3}
    \MS{encode}&_{\Sigma + \Tau + \Bool}&(\inl x) &:= \inr \false \cons \map~(\inl \circ \inr)~\MS{encode}(x) \\
    \MS{encode}&_{\Sigma + \Tau + \Bool}&(\inl y) &:= \inr \true  \cons \map~(\inl \circ \inl)~\MS{encode}(y) \\
    \MS{encode}&_{\Sigma + \Bool}& (\nil)         &:= \inr \false \\
    \MS{encode}&_{\Sigma + \Bool}& (x :: xs)      &:= \inr \true \cons \map~\inl~\MS{encode}_{\Sigma}(x) \app \MS{encode}_{\Sigma + \Bool}(xs) \\
    \MS{encode}&_{\Sigma+\Tau}   &(x,y)           &:= \map~\inl~\MS{encode}_{\Sigma}(x) \app \map~\inr~\MS{encode}_{\Tau}(y) \\
    \MS{encode}&_{\Bool}         &(0)             &:= [\false] \\
    \MS{encode}&_{\Bool}         &(\MS S~n)       &:= \true \cons \MS{encode}_{\Bool}(n)
  \end{alignat*}
\end{frame}

\begin{frame}{$n$-lift}
  \begin{figure}
    \center%
    \begin{tikzpicture}
      \begin{scope}
        \node(a0)[yshift=-0 cm]  {$0$};
        \node(a1)[yshift=-1 cm]  {$1$};
        \node(a2)[yshift=-2 cm]  {$2$};
      \end{scope}
      \begin{scope}[xshift=4cm, yshift=0cm]
        \node(b0)[yshift=-0 cm] {$0$};
        \node(b1)[yshift=-1 cm] {$1$};
        \node(b2)[yshift=-2 cm] {$2$};
        \node(b3)[yshift=-3 cm] {$3$};
        \node(b4)[yshift=-4 cm] {$4$};
      \end{scope}
      \begin{scope}[xshift=5cm, yshift=-3cm]
        \node(none)             {$\None$};
      \end{scope}
      \path (a0) edge[<->] (b1);
      \path (a1) edge[<->] (b0);
      \path (a2) edge[<->] (b3);
      \path (b2) edge[-> ] (none);
      \path (b4) edge[-> ] (none);
    \end{tikzpicture}
    \caption{Tape retract encoded as the index-vector $[1, 0, 3]$.}
  \end{figure}
\end{frame}


\begin{frame}{$\Sigma$-lift}
  \begin{figure}
    \center%
    \begin{tikzpicture}
      \begin{scope}
        \node(alpha)[yshift=-0 cm] {$\alpha$};
        \node(beta) [yshift=-1 cm] {$\beta$};
        \node(gamma)[yshift=-2 cm] {$\gamma$};
      \end{scope}
      \begin{scope}[xshift=4cm, yshift=1cm]
        % Symbols of $\Tau$
        \node(a)[yshift=-0 cm] {$a$};
        \node(b)[yshift=-1 cm] {$b$};
        \node(c)[yshift=-2 cm] {$c$};
        \node(d)[yshift=-3 cm] {$d$};
        \node(e)[yshift=-4 cm] {$e$};
      \end{scope}
      \path (alpha) edge[<->] (a);
      \path (beta)  edge[<->] (b);
      \path (gamma) edge[<->] (d);
      \path (gamma) edge[<-, dotted] (c);
      \path (gamma) edge[<-, dotted] (e);
    \end{tikzpicture}
    \caption{Example alphabet retract; $\gamma$ is the default symbol in $\Sigma$.}
  \end{figure}
\end{frame}

\begin{frame}{Basic machines}
  Classes of $1$-tape machines for each basic action \\(together with partitioning functions):
  \begin{itemize}
    \item Move head in the direction $D$
    \item Write one symbol
    \item Read one symbol
    \item nop
    \pause%
    \item Write a list of symbols (defined per recursion over the string)
  \end{itemize}
\end{frame}

\begin{frame}{Coq code dimensions}
	% \tiny \verbatiminput{../wc.txt}
	\begin{table}
		\begin{tabular}{|l|r|r|}
			\hline
			& Lines Spec & Lines Proof\tabularnewline
			\hline
			Prelim & 235+ & 286+\tabularnewline
			Relations/Retracts & 475 & 269\tabularnewline
			Def. $n$TM & 215 & 85\tabularnewline
			$\Sigma$-Lift & 132 & 129\tabularnewline
			$n$-Lift & 153 & 195\tabularnewline
			Basic machines & 142 & 69\tabularnewline
			Encoding & 170 & 164\tabularnewline
			Computation & 241 & 289\tabularnewline
			Combinators & 158 & 354\tabularnewline
			Finite and $\Bool$ TM & 112 & 203\tabularnewline
			\hline
			$\Sigma$ & 2033 & 2042\tabularnewline
			\hline
		\end{tabular}
	\end{table}
\end{frame}

\end{document}

% vim: ts=2 sts=2 sw=2 expandtab textwidth=150
