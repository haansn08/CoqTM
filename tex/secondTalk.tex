\documentclass{beamer} %
% \usetheme{CambridgeUS}
\usepackage[latin1]{inputenc}
% \usefonttheme{professionalfonts}
\usepackage{tikz}
\usepackage{amsmath}
\usepackage{verbatim}
\usepackage{listings}
\usepackage{minted}
\usecolortheme{orchid}
\usetikzlibrary{arrows,shapes,snakes}


%%
%% Important note:
%% Add [fragile] to each frame using listings!
%%

%%%
%%% Shared preamble for all files, e.g. thesis, TikZ standalones, slides, etc.
%%% It defines \macros for types, Turing machines, etc.
%%%

% Packages needed
\usepackage[utf8]{inputenc}
\usepackage{geometry}
\usepackage[small,compact]{titlesec}
\usepackage[final]{listings}
\usepackage{amsmath}
\usepackage[amsmath,hyperref,thmmarks]{ntheorem}
% Warning: The package ntheorem defines a \None macro!
\usepackage{amssymb}
\usepackage{tipa}
\usepackage[english]{babel}
\usepackage{lstautogobble}
\usepackage{proof}
\usepackage{bussproofs}
\usepackage{xparse}
\usepackage{needspace}
\usepackage{xspace}
\usepackage{mathpartir}
\usepackage{stmaryrd} % for |llbracket and \rrbracket
\usepackage{standalone} % useful to out-source graphics


% TikZ ist *kein* Zeichenprogramm.
\usepackage{tikz}
\usetikzlibrary{arrows,shapes,snakes,automata,backgrounds,fit,positioning}
\usepackage{tikz-cd} % for commutative diagrams


%% Formating
\newcommand{\MS}[1]{\ensuremath{\mathsf{#1}}}
\newcommand{\MST}[1]{${\mathsf{#1}}$}
\newcommand{\IsMathMode}{\ifmmode{This is math mode}\else{This is not math mode}\fi}

%% Logic symbols
\newcommand{\defop}{\mathop{:=}}
\newcommand{\imp}{\mathbin{\rightarrow}~}
\newcommand{\Imp}{\mathbin{\Rightarrow}~}
\renewcommand{\iff}{\mathbin{\leftrightarrow}}


% ++ operator:
% Source: https://tex.stackexchange.com/questions/4194/how-to-typeset-haskell-operator-and-friends
\newcommand\doubleplus{+\kern-1.3ex+\kern0.8ex}
\newcommand\mdoubleplus{\ensuremath{\mathbin{+\mkern-10mu+}}}
\newcommand{\app}{\mdoubleplus}

\newcommand{\rew}{\Rightarrow}
\newcommand{\trew}{\stackrel{\textrm{T}}\Rightarrow}
\newcommand{\llrew}{\stackrel{\textrm{L}}\Rightarrow}
\newcommand{\rlrew}{\stackrel{\textrm{R}}\Rightarrow}
\newcommand{\arew}{\triangleright}
\newcommand{\conc}{\mathop{{+}\hskip-5pt{+}}}
\newcommand{\gen}{\Rightarrow}

%% Sets
% \newcommand{\lam}[2]{\lambda#1{.}\hskip.7pt#2}
\newcommand{\setOf}[1]{\bigl\{ #1 \bigr \}}
\newcommand{\setMap}[2]{\setOf{#1~\big|~#2}}
\newcommand{\depPair}[2]{\setOf{#1~{\&}~#2}}
\newcommand{\pair}[2]{\bigl( #1 , #2 \bigr)}
\newcommand{\class}[1]{\bigl[ #1 \bigr]}
\newcommand{\choice}[1]{\bigl< #1 \bigr>}
\newcommand{\explainRel}[2]{\stackrel{\text{#1}}{#2}}
\newcommand{\family}[2]{\bigl( #1 \bigr)_{#2}}
\newcommand{\from}{:}
\renewcommand{\to}{\rightarrow}

%% Types
\newcommand{\Bool}{\mathbb{B}}
\newcommand{\Fin}{\mathbb{F}}
\newcommand{\Nat}{\mathbb{N}}
\newcommand{\Prop}{\mathbb{P}}
\newcommand{\Type}{\mathbb{T}}
\newcommand{\Unit}{\MS{1}}
\newcommand{\Option}{\mathcal{O}}
\newcommand{\List}{\mathcal{L}}
\newcommand{\Rel}{\MS{Rel}}

\newcommand{\True}{\top}
\newcommand{\False}{\bot}

%% Tapes
\newcommand{\tape}[1]{[ #1 ]}
\newcommand{\tapePointer}[1]{\underset{\uparrow}{#1}}
\newcommand{\niltape}{\tape{\tapePointer{}}}
\newcommand{\midtape}[3]{\tape{#1~\tapePointer{#2}~#3}}
\newcommand{\leftof}[2]{\tape{\tapePointer{}~#1~#2}}
\newcommand{\rightof}[2]{\tape{#1~#2~\tapePointer{}}}

% \newcommand{\niltape}{\MS{niltape}}
% \newcommand{\midtape}[3]{\MS{midtape}~#1~#2~#3}
% \newcommand{\leftof}[2]{\MS{leftof}~#1~#2}
% \newcommand{\rightof}[2]{\MS{rightof}~#1~#2}

%% Turing machine types
\newcommand{\Loop}{\MS{loop}}
\newcommand{\Tape}{\MS{Tape}}
\newcommand{\Tapes}[1]{\Tape^{#1}}
\newcommand{\TM}{\MS{TM}}
\newcommand{\Move}{\MS{Move}}
\newcommand{\Act}{\MS{Act}}
\newcommand{\Conf}{\MS{Conf}}
\newcommand{\Tau}{\Gamma}

%% Relations
\newcommand{\rif}{\mathbin{\phi}}
\newcommand{\at} [2][]{#1{|}_{#2}}
\newcommand{\att}[2][]{#1{|\mkern-1.5mu|}_{#2}}
\DeclareMathOperator{\ignoreParam}{\Uparrow}
\DeclareMathOperator{\hideParam}{\Downarrow}


%% Constructors
\DeclareMathOperator{\inl}{\ensuremath{\MS{inl}}}
\DeclareMathOperator{\inr}{\ensuremath{\MS{inr}}}
\newcommand{\Some}[1]{\left\lfloor {#1} \right\rfloor}
% \None is defined sometimes
\renewcommand{\None}{\emptyset}
\newcommand{\true}{\MS{true}}
\newcommand{\false}{\MS{false}}
\newcommand{\unit}{\MS{()}}
\newcommand{\nil}{\MS{nil}}
\newcommand{\cons}{\mathbin{::}}

%% Functions
\newcommand{\map}[2]{\ensuremath{\MS{map}~#1~#2}}
\newcommand{\maptwo}[3]{\ensuremath{\MS{map}_2~#1~#2~#3}}
\newcommand{\rev}[1]{\MS{rev}~#1}

%% Vector
\newcommand{\Vector}[1]{\left[ #1 \right]}
\DeclareMathOperator{\hd}{\ensuremath{\MS{hd}}}
\DeclareMathOperator{\tl}{\ensuremath{\MS{tl}}}
\newcommand{\length}[1]{\left| #1 \right|}
\newcommand{\blength}[1]{\bigl| #1 \bigr|}


%%
%% Encding
%%
\newcommand{\contains}{\simeq}
\newcommand{\size}[1]{\length{encode(#1)}}

%% Semantics
\newcommand{\terminates}{\mathrel{\triangleright}}
\newcommand{\TerminatesIn}{\mathrel{\downarrow}}
\newcommand{\Realise}{\mathrel{\vDash}}
\newcommand{\RealiseIn}[1]{\mathrel{\vDash^{#1}}}

%%
%% Turing Machines
%%

%% Control flow operators
\newcommand{\While}{\MS{While}}
\newcommand{\Seq}{;~}
\newcommand{\Match}{\MS{Match}}
\newcommand{\If}[3]{\MS{If}~#1~\MS{Then}~#2~\MS{Else}~#3}
\newcommand{\Let}[2]{\MS{let}~#1~\MS{in}~#2}
\newcommand{\cond}[3]{\MS{if}~#1~\MS{then}~#2~\MS{else}~#3}
\newcommand{\Nop}{\MS{Nop}}
\newcommand{\Return}[2]{\MS{Return}~#1~#2}
% \newcommand{\Return}[2]{\MS{Return}_{#2}~#1}

%% Lifts
% #1 is the machine, #2 the lifting
\newcommand{\LiftTapes}[2]{\mathop{\Uparrow_{#2}} #1}
\newcommand{\LiftAlphabet}[2]{\mathop{\Uparrow_{#2}} #1}
% #1 is the machine, #2 the alphabet lifting, and #3 the tape-lifting
\newcommand{\LiftBoth}[3]{\mathop{\Uparrow_{#2;~#3}} #1}




%%%
%%% lstlisting
%%%

% Style and language to define complex multi-line definitions similar to Coq code
\lstdefinelanguage{semicoq}{
  keywords={if,then,else,true,false,match,Match,If,Then,Else,Nop,Return,Move,Reset,DoAct,WriteMove,L,R,N},
  comment=[s]{(*}{*)},
}

%% Overlap #2 over phantom #1, e.g.
%% % XX\phalign{abcdefg}{YY}XX \\
%% % XXabcdefgXX
%% gets
%% XXYY     XX
%% XXabcdefgXX
%% Idea from https://tex.stackexchange.com/questions/212710/fill-space-created-by-phantom-with-other-text
\newcommand{\phalign}[2]{\makebox[0pt][l]{\ensuremath{#2}}\phantom{#1}}

\lstdefinestyle{semicoqstyle}{
  mathescape=true,
  keywordstyle=\textsf,
  language=semicoq,
  literate={
    {=>}{{$\Rightarrow$}}2
    {>->}{{$\rightarrowtail\,$}}2
    {<->}{{$\leftrightarrow$ }}2
    {->}{{$\to$ }}3
    {~}{{$\lnot$}}1
    {/\\}{{$\land$}}2
    {\\/}{{$\lor$}}2
    {forall}{{$\forall$}}1
    {exists}{{$\exists$}}1
    {<>}{{$\not =$}}{1}
    {<=}{{$\leq$}}{1}
    {<}{{$\lt$}}{1}
    {>=}{{$\ge$}}{1}
    {>}{{$\gt$}}{1}
    {[}{{$[$}}{1}
    {|}{{$|$}}{1}
    {]}{{$]$}}{1}
    {])}{{$])$}}{2}
    {(}{{$($}}{1}
    {)}{{$)$}}{1}
    {match}{{$\MS{match}$}}5
    {if}{{$\MS{if}$}}1
    {then}{{$\phalign{\MS{else}}{{\MS{then}}}$}}3
    {else}{{$\phalign{\MS{else}}{{\MS{else}}}$}}3
    {If}{{$\MS{If}$}}2
    {Then}{{$\phalign{\MS{Else}}{{\MS{Then}}}$}}4
    {Else}{{$\phalign{\MS{Else}}{{\MS{Else}}}$}}4
  }
}

\lstdefinelanguage{pseudocode}{
  keywords={If,Then,Else,Do,While,Reset,Return,Continue,Break},
}

\lstdefinestyle{pseudocode}{
  mathescape=true,
  language=pseudocode,
  literate={
    {:=}{{$\leftarrow$}}{2}
    {<>}{{$\not =$}}{1}
    {<=}{{$\leq$}}{1}
    {<}{{$\lt$}}{1}
    {>=}{{$\ge$}}{1}
    {>}{{$\gt$}}{1}
  }
}






%%% Local Variables:
%%% mode: LaTeX
%%% TeX-master: "thesis"
%%% End:


%% Use these versions of the type names
\renewcommand{\Type}{\mathbb{T}}
\renewcommand{\Nat}{\mathbb{N}}
\renewcommand{\Bool}{\mathbb{B}}
\renewcommand{\Option}{\mathcal{O}}
\renewcommand{\List}{\mathcal{L}}
\newcommand{\Pow}{\mathcal{P}}

% Hide Controls
\beamertemplatenavigationsymbolsempty%
% \setbeamertemplate{footline}[page number]
%% Show only current page number
\setbeamertemplate{footline}{\raisebox{5pt}{\makebox[\paperwidth]{\hfill\makebox[20pt]{\scriptsize{\color{gray}\insertframenumber}}}}}

%% Display a list of references at the bottom
\newcommand\refs[1]{%
  \begin{textblock*}{8cm}(0.3cm,9.0cm)%
    \scriptsize {\color{gray}#1}
  \end{textblock*}
}


\title{Formalising Multi-Tape Turing Machines In Coq}
\subtitle{Second Bachelor Seminar Talk}
\author{Maximilian Wuttke}
\institute{Saarland University\\
  \bigskip
  \tiny
  Programming Systems Lab
}
\date{March 20, 2018\\
  \bigskip{}
  \bigskip{}
  {
    \tiny
    Advisor: Yannick Forster\\
    Supervisor: Prof.\ Dr.\ Gert Smolka%
  }
}

\begin{document}


\tikzstyle{every picture}+=[remember picture]




\frame{\titlepage}

\begin{frame}{Small Recap: Muli-Tape Turing Machines in Coq}

  \begin{itemize}
  \item \textbf{Partitioned machines}: $pM : \setOf{M : \MS{mTM}~\Sigma~n;~ f \from Q_{M} \to F}$ for \textbf{finite} types $F$
  \item Parametrised relations: $R \subseteq (\Tape_\Sigma^n) \times (F \times \Tape_\Sigma^n)$
  \item Weak realisation: $pM \VDash R$
  \item Termination: $M \downarrow T$
  \end{itemize}

\end{frame}

\begin{frame}{Realisation}

  \begin{definition}[Realisation, $pM \VDash R$]
    Let $R \subseteq (\Tape_\Sigma^n) \times (F \times \Tape_\Sigma^n)$ for a finite $F$.
    \begin{multline*}
      \mathbf{pM \VDash R} :=
      \forall (t_{in}~t_{out} : \Tape_\Sigma^n)~(q_{out}:Q_M)~(k:\Nat). \\
      M(t_{in}) \triangleright^k (q_{out}, t_{out}) \rightarrow
      \left(t_{in}, \left(f(q_{out}), t_{out} \right) \right) \in R.
    \end{multline*}
    We say that $pM$ \textbf{realises} $R$.
  \end{definition}

  \begin{lemma}[Monoticity of $\VDash$]
    If $pM \VDash R_1$ and $R_1 \subseteq R_2$, then $pM \VDash R_2$.
  \end{lemma}

\end{frame}

\begin{frame}{Termination}

  \begin{definition}[Termination, $M \downarrow T$]
    Let $T \subseteq (\Tape_\Sigma^n) \times \Nat$.
    \[
      \mathbf{M \downarrow T} :=
      \forall (t_{in},~k) \in T.~
      \exists c_{out},~M(t_{in}) \triangleright^k c_{out}
    \]
    We say that $M$ \textbf{terminates in} $T$.
  \end{definition}

  \begin{lemma}[Monoticity of $\downarrow$]
    If $M \downarrow T_1$ and $T_2 \subseteq T_1$, then $M \downarrow T_2$.
  \end{lemma}

\end{frame}


\begin{frame}{Combinators for Parametrised Machines}
  \emph{Shallow-embedded} language for constructing partitioned machines:
  
  \begin{itemize}
  \item $\MS{Match}~pM_1~(\lam {x:F_1} \cdots pM_2 \cdots pM_3 \cdots)$
  \item $\MS{If}~pM_1~pM_2~pM_3$
  \item $pM_1;~ pM_2$
  \item $\MS{While}~pM$
  \end{itemize}

  \pause

  \begin{lemma}[Correctness of $\MS{While}$]
    Let $pM$ be a partitioned (and $R$ parametrised) over $\Bool \times F$.\\
    % Overload While for a relation operator and a machine operator?
    If $pM \VDash R$, then $\MS{While}~pM \VDash \MS{While}~R$ with
    \[
      \MS{While}~R :=
      (\bigcup_{y \in F} \bigl({R\at{(true, y)}}\bigr)^* \circ R\at{\MS{fst}=\false}.
    \]
  \end{lemma}
\end{frame}

\begin{frame}{Machine transformations}
  \textbf{Problem:}
  When combining machines, the numbers of tapes and the alphabet have to match!
  \pause%
  \bigskip

  \textbf{Solution:} Two operations on machines:
  \begin{itemize}
    \item $n$-Lift: add/rearange tapes
    \item $\Sigma$-Lift: translate symbols
  \end{itemize}
\end{frame}
\begin{frame}{$n$-Lift}
  Let $f \from \Fin_m \hookrightarrow \Fin_n$ be an injection between tape indexes.

  Let $M$ be an $m$-tape $F$-partitioned Turing machine parametrised over $\Sigma$,
  and $R$ be a $F$-parametrised relation.

  Define an $n$-tape Turing machine $\MS{Lift}_f~M$ and a relation $\MS{Lift}_f~R$.
  \begin{align*}
    \MS{Lift}_f~R := \setMap{(t_{in}, (a, t_{out})) &}{ \bigl(f^{-1}(t_{in}), (a, f^{-1}(t_{out})) \bigr) \in R } \cap \\
    \Uparrow \setMap{(t_{in}, t_{out}) &}{ \forall i \notin \MS{img}~f.~ t_{in}[i] = t_{out}[i] }
  \end{align*}
  {\footnotesize (Where $f^{-1} \from \Tape^m \to \Tapes^n$)}
  % (Tapes that are not in the image of $f$ do nothing, the other tapes simulate the corresponding tapes of $M$.)

  \begin{lemma}[Correctness of the $n$-Lift]
    \[
      M \VDash R \rightarrow (\MS{Lift}_f~M) \VDash (\MS{Lift}_f~R)
    \]
  \end{lemma}

\end{frame}

\begin{frame}{$\Sigma$-Lift}
  Let $f \from \Sigma \hookrightarrow \Tau$ be an injection between alphabets.

  Let $\mathbf{default} : \Sigma^n$.

  Let $M$ be an $m$-tape $F$-partitioned Turing machine parametrised over $\Sigma$,
  and $R$ be a $F$-parametrised relation.

  Define an $n$-tape Turing machine $\MS{Lift}_{f,\MS{default}}~M$ and a relation $\MS{Lift}_{f,\MS{default}}~R$.
  \begin{align*}
    \MS{Lift}_{f,\MS{default}}~R := \setMap{(t_{in}, (a, t_{out}))}{ (f^{-1}_{\MS{default}} (t_{in}) , (a, f^{-1}_{\MS{default}} (t_{out}))) \in R}
  \end{align*}

  \begin{lemma}[Correctness of the $\Sigma$-Lift]
    \[
      M \VDash R \rightarrow (\MS{Lift}_{f, \MS{default}}~M) \VDash (\MS{Lift}_{f, \MS{default}}~R)
    \]
  \end{lemma}
\end{frame}


\begin{frame}[fragile]{Encoding Values on Tapes}

  \begin{itemize}
  \item
    Typeclass for \emph{encodable} types $X$ and alphabets $\Sigma$:

    \footnotesize
    \begin{minted}{coq}
Class encodable (X : Type) (sig : finType) := {
  encode : X -> list sig;
}.
    \end{minted}
    For example: $\MS{encode}~(b:\Bool) := [b]$ and $\MS{encode} (n:\Nat) := 1^n 0$.
    \pause

  \item
    Insert start end end symbols into an encoding:
    $\Sigma^+ := \Bool + \Sigma$. \pause
    
  \item
    Define class of predicates for \emph{calue-containing} of tapes:

    \begin{definition}[value-containing, $t \eqsim_{[r1;r2]} x$]
      Let $X$ be encodable over $\Sigma$, and $t \in \Tape_\Sigma$.
      \[
        t \eqsim_{[r1;r2]} x :=
        t = \midtape{(r_1 \app [\inl \true])}{(\inr y)}{(\map~\inr~ys \app \inl \false \cons r_2)}
        }.
      \]
      for $\MS{encode}(x) = y \cons ys$.
      % or:
      % \[
      %   t \eqsim_{[r1;r2]} x :=
      %   t = \midtape{(r_1 \app [\inl \true])}{\inl \false }{r_2}
      % \]
      % f
      % and $\MS{encode}(x) = \nil$ and
    \end{definition}

    { \footnotesize
      We write $t \eqsim x$ for $\exists r_1~r_2.~t \eqsim_{[r_1;r_2]} x$ \\
      and $t \eqsim_{\{;s\}} x$ for $\exists r_1~r_2.~t\eqsim_{[r_1;r_2]} x \land |r_2| \le s$.
    }
  \end{itemize}
  
\end{frame}

\begin{frame}[fragile]{A Relation for (Unary) Function Computation}

  Convention:
  \begin{itemize}
  \item Input tapes; output tape; ``internal tapes''
  \item Callee-safe
  \item ``Internal tapes'' stay right
  \item Output grows to the left
  \end{itemize}
  \pause
  { \footnotesize
    \begin{columns}
      \begin{column}{0.5\textwidth}
        Let $X, Y, Z$ be encodable over $\Sigma$.
        \begin{alignat*}{2}
          & \MS{Computes}~(f:X \to Y) := \\
          & \quad \lam {t_{in}~(\_, t_{out})} \forall (x:X), \\
          & \quad\quad t_{in}[0] \eqsim x \rightarrow \\
          & \quad\quad \MS{isRight}~t_{in}[1] \rightarrow \\
          & \quad\quad (\forall i : \Fin_n.~\MS{isRight}~t_{in}[2+i]) \rightarrow \\
          & \quad\quad t_{out}[0] \eqsim x ~\land \\
          & \quad\quad t_{out}[1] \eqsim_{\{;0\}} f~x ~\land \\
          & \quad\quad (\forall i : \Fin_n.~\MS{isRight}~t_{out}[2+i])
        \end{alignat*}
      \end{column}
      \pause
      \begin{column}{0.5\textwidth}
        \begin{alignat*}{2}
          & \MS{Computes2}~(f:X \to Y \to Z) := \\
          & \quad \lam {t_{in}~(\_, t_{out})} \forall (x:X)~(y:Y), \\
          & \quad\quad t_{in}[0] \eqsim x \rightarrow \\
          & \quad\quad t_{in}[1] \eqsim y \rightarrow \\
          & \quad\quad \MS{isRight}~t_{in}[2] \rightarrow \\
          & \quad\quad (\forall i : \Fin_n.~\MS{isRight}~t_{in}[3+i]) \rightarrow \\
          & \quad\quad t_{out}[0] \eqsim x ~\land \\
          & \quad\quad t_{out}[1] \eqsim y ~\land \\
          & \quad\quad t_{out}[1] \eqsim_{\{;0\}} f~x~y ~\land \\
          & \quad\quad (\forall i : \Fin_n.~\MS{isRight}~t_{out}[3+i])
        \end{alignat*}
      \end{column}
    \end{columns}
  }
\end{frame}


\begin{frame}{Value-Manipulating Machines}
  \begin{itemize}
  \item Match/destruct
  \item Constructor
  \item CopyValue
  \item MoveRight
  \item WriteValue
  \end{itemize}
\end{frame}


\begin{frame}[fragile] \frametitle{Match $\Nat$}
  \footnotesize

  {\footnotesize
    \begin{align*}
      & \MS{Match\_Nat\_Rel} := \\
      & \quad \lam {t_{in}~(y_{out}, t_{out})} \forall (n:\Nat) \uncover<2->{(s:\Nat)}, \\
      & \quad\quad t_{in}[0] \eqsim_{\uncover<2->{\{;s\}}} n \rightarrow \\
      & \quad\quad \MS{match}~n~\MS{with} \\
      & \quad\quad |~0 \Rightarrow t_{out} = t_{in} ~\land~ y_{out} = \false \\
      & \quad\quad |~S~n' \Rightarrow t_{out}[0]  \eqsim_{\uncover<2->{\{;s\}}} n' ~\land~ y_{out} = \true \\
      & \quad\quad \MS{end}
    \end{align*}
  }

  \pause\pause
  \begin{columns}
    \begin{column}{0.5\textwidth}
      
\begin{lstlisting}[language=c, escapechar=!]
if (n--) {
  // ...
} else {!\tikz\node [coordinate] (n1) {};!
  // ...
}
\end{lstlisting}
    \end{column}
    \begin{column}{0.5\textwidth}
      \begin{align*}
        & \MS{If}~(\MS{Inject}~\MS{MatchNat}~\Vector{ \text{\tiny \# of tape where $n$ is stored} }) \\
        & \quad pM_1~pM_2
      \end{align*}
    \end{column}
  \end{columns}


  \begin{tikzpicture}[overlay]
    % \path[->]<1-> (n1) edge [bend left] (n2);
    \draw[->, snake] ([xshift=1.5cm] n1) -- +(2cm,0);
  \end{tikzpicture} 
\end{frame}

\end{document}