\chapter{Definitions}
\label{chap:definitions}

In this chapter, we will formally define the notion of multi-tape Turing machines.

We took the definitions of multi-tape Turing machines and their tapes from \cite{asperti2015}.

We will introduce our notions for specifying the semantics of machines.

We will define some basic machines, state and proof their correctness, using the definitions and lemmas developed in this chapter.

\section{Machines and Tapes}
\label{sec:machine_tapes}


\begin{definition}[Movement]
  \label{def:movement}
  There are three possible movements:
  $$\MS{Move} ::= L ~|~ R ~|~ N.$$
\end{definition}


\begin{definition}[Multi-tape Turing machine]
  \label{def:mTM}
  An \emph{$n$-tape Turing machine} over a finite alphabet $\Sigma$ is a record $M = (Q, \delta, s, h)$ where
  \begin{itemize}
  \item $Q$ is a finite type
  \item $\delta \from Q \times (\Option~\Sigma)^n \to Q \times (\Option~\Sigma \times \MS{Move})^n$
  \item $start:Q$
  \item $halt \from Q \to \Bool$ 
  \end{itemize}

\end{definition}

The transition function $\delta$ yields for every state and vector of $n$ read symbols yields the new state and a vector of $n$ symbols to write and a
direction to move.  The read symbols are optional, since it can be the case that there is no symbol under the head of a tape. $start$ is the start state
of the machine and $halt$ represents the subset of halting states.  Tuples of the type $\Option(\Sigma) \times \Move$ are called \emph{actions}.  They
are refered to with the symbol $\MS{Act}_\Sigma$ or $\MS{Act}$ if $\Sigma$ is clear.

At this moment, we make no distinction between input and output tapes.  Note that the transition function has to yield a successor state even on
halting states.  Also note that while we parameterized Definition~\ref{def:mTM} over the alphabet $\Sigma$ and the number of tapes $n$, but we
``abstract'' the states.  Our machines behave deterministically, since $\delta$ is a function.  We write $\MS{TM}_\Sigma^n$ for the type of $n$-tape
Turing machines over the alphabet $\Sigma$.

Later, when we want to verify complex machines, we do not want to reason about internal states.  We rather want to reason about partition of states,
e.g. positive or negative states.  In general, if $F$ is a finite type, then
$\MS{TM}_\Sigma^n(F) := \left\{ M:\MS{TM}_\Sigma^n ~\&~ Q_M \to F \right\}$ is the sigma type of a machine and a partitioning function from the states
to $F$.  For a machine $M$ together with a partition function $p$, we write $(M; p)$.  We often use the symbol $M$ for parametrised machines.  It
should be clear from the context, wether $M$ is a partitioned or unpartitioned machine.

We also use the definition of \emph{tapes} from the Matita development of Asperti et~al~\cite{asperti2015}. Arbitrary much memory can be allocated on
the tape, however every tape has only finitely many symbols, i.e. there is a left-most and a right-most symbol.  A tape essentially is a triple
$(ls,m,rs)$, where the symbol $m$ is the symbol on which the (read/write) head of tape is.  It is essential, that the symbol lists ordered such, that
the head of the list is the symbol next to the symbol $m$.  With other words, $ls$ is stored in reversed order.

There are three cases where there is no current symbol: the tape can be completely empty, or the head can be to the left (or right) outermost of the
tape.  Formally, tapes are defined inductively:

\begin{definition}[Tape]
  \label{def:tape}
  Let $\Sigma$ be a finite alphabet.  Then $\Tape_\Sigma$ is defined as the inductive type:
  \begin{align*}
    & \Tape_\Sigma := \\
    & \quad | \quad \MS{niltape} \\
    & \quad | \quad \MS{leftof}  ~ (r:\Sigma) ~ (rs:\List(\Sigma)) \\
    & \quad | \quad \MS{midtape} ~ (ls:\List(\Sigma)) ~ (m:\Sigma) ~ (rs:\List(\Sigma)) \\
    & \quad | \quad \MS{rightof} ~ (l:\Sigma) ~ (ls:\List(\Sigma)).
  \end{align*}
\end{definition}

We introduce a informal notation of tapes, where the symbols are represented from left to right, hence we have to reverse the lists $ls$.  The
position of the head is marked by the arrow:
\begin{align*}
  \niltape &:= \MS{niltape}\\
  \leftof{r}{rs} &:= \MS{leftof}~r~rs\\
  \midtape{ls}{m}{rs} &:= \MS{midtape}~(\rev~rs)~m~rs\\
  \rightof{ls}{l} &:= \MS{rightof}~(\rev~ls)~l
\end{align*}


Now we can define the \emph{configuration} of a multi-tape Turing machine.  It is captured by the current state and the vector of the $n$ tapes:
\begin{definition}[Configuration]
  \label{def:config}
  A \emph{configuration} of an $n$-tape Turing machine $T$ over the alphabet $\Sigma$ is a record $\MS{Conf} := \{\MS{state}:Q_M; \MS{tapes}:\Tape_\Sigma^n\}$.
\end{definition}

Now we define tape movement and how to write symbols on tapes.
\begin{alignat*}{2}
  \MS{mv}_R&~(\leftof{r}{R}               &&:= \midtape{\nil}{r}{R} \\
  \MS{mv}_R&~(\midtape{L}{m}{\nil})       &&:= \rightof{L}{m} \\
  \MS{mv}_R&~(\midtape{L}{m}{r \cons R)}) &&:= \midtape{L \app [m]}{r}{R} \\
  \MS{mv}_N&~(t)                          &&:= t
\end{alignat*}
The function $\MS{mv}_L$ is defined analogously.

To define the function $\MS{write} \from \Option(\Sigma) \to \Tape \to \Tape$, we first need functions
$\MS{left},~\MS{right} \from \Tape \to \List(\Sigma)$, that return the symbols to the left (or right) side of the pointer.
\begin{alignat*}{4}
  \MS{left} &~(\MS{niltape})                 &&:= \nil
  \quad\quad\quad\quad
  & \MS{right}&~(\MS{niltape})               &&:= \nil \\
  \MS{left} &~(\MS{leftof}~{r}~{rs})         &&:= \nil
  & \MS{right}&~(\MS{leftof}~{r}~{rs})       &&:= r \cons rs \\
  \MS{left} &~(\MS{midtape}~{ls}~{m}~{rs})   &&:= ls
  & \MS{right}&~(\MS{midtape}~{ls}~{m}~{rs}) &&:= rs \\
  \MS{left} &~(\MS{rightof}~{ls}~{l})        &&:= l \cons ls
  & \MS{right}&~(\MS{rightof}~{ls}~{l})      &&:= \nil
\end{alignat*}

Note that as a consequence of the informal notation, we have
$$\MS{left}(\midtape{ls}{m}{rs}) = \MS{left}(\MS{midtape}~(\rev{ls})~{m}~{rs}) = \rev{ls}.$$

Now we can define the function $\MS{wr} \from \Tape \to \Option(\Move) \to \Tape$, that writes an optional symbol to a tape.  When we write $\None$,
the tape remains unchanged.  But if we write $\Some a$, we get a $\MS{midtape}$, where the left and right symbols remain unchanged and $a$ is now in
the middle.
\begin{alignat*}{3}
  \MS{wr}~t &~ \None   &&:= t \\
  \MS{wr}~t &~ \Some a &&:= \MS{midtape}~(\MS{left}~t)~{a}~(\MS{right}~t)
\end{alignat*}
To define the function $\MS{step} \from \MS{Conf} \to \MS{Conf}$, we need to know the symbols on the tapes.
Therefore we define a function $\MS{current} \from Tape \to \Option(\Sigma)$.
It returns $\None$ if the pointer is not under a symbol, and $\Some a$ if the pointer is under the symbol $a$.
\begin{alignat*}{2}
  \MS{current}&~(\MS{midtape}~{ls}~{m}~{rs})&&:= \Some m \\
  \MS{current}&~\_                          &&:= \None
\end{alignat*}
We can state a correctness lemma of the function $\MS{write}$:
\begin{lemma}[Write]
  \label{lem:write}
  For all tapes $t$ and symbols $\sigma:\Sigma$:
  % TODO: Align it, for example like in https://tex.stackexchange.com/questions/12771/mix-align-and-enumerate
  \begin{enumerate}
  \item $\MS{right}   (\MS{wr}~t~\Some\sigma) = \MS{right}(t)$
  \item $\MS{left}    (\MS{wr}~t~\Some\sigma) = \MS{left} (t)$
  \item $\MS{current} (\MS{wr}~t~\Some\sigma) = \Some\sigma$
  \end{enumerate}
\end{lemma}
\begin{proof}
  All claims follow by case analysis over $t$.
\end{proof}

We can now define the function $\MS{step} \from \MS{Conf} \to \MS{Conf}$.  First the machine reads all the currents symbols from the tapes.  It
inserts this vector and the current state into the transition function $\delta$.  Then, each tape writes the symbol and moves its head into the
direction, which $\delta$ yielded for it.  The machine ends up in a new step $q'$.

\begin{definition}[Step]
  \label{def:step}
  \begin{alignat*}{2}
    \MS{wr\_mv}&~t~(x, d)      &~:=~& \MS{mv}_d (\MS{wr}~t~x) \\
    \MS{step}&~(q, \MS{tapes}) &~:=~& \Let{(q', \MS{actions}) := \delta(q, \map{\MS{current}}{\MS{tapes}})}{ \\
      &                        &~  ~& (q', \maptwo{\MS{wr\_mv}}{\MS{tapes}}{\MS{actions}})}
  \end{alignat*}
\end{definition}

To define the execution of a machine, we first introduce an abstract recursive \emph{loop} function of the type
$\MS{loop} \from \Nat \to (A \to A) \to (A \to \Bool) \to A \to \Option(A)$, for every $A:\Type$:
\begin{align*}
  \MS{loop}~n~f~h~s :=
  \begin{cases}
    \Some{s}              & h(s) = \true \\
    \None                 & h(s) = \false \land n = 0 \\
    \MS{loop}~(n-1)~f~h~(f s)  & h(s) = \false \land n > 0
  \end{cases}
\end{align*}

We can show some basic lemmas about $\MS{loop}$.
\begin{lemma}[Basic facts about $\MS{loop}$]
  \label{lem:loop}
  Let $f \from A \to A$ be a step function and $h \from A \to \Bool$ be a halting function, $s:A$, and $k, l:\Nat$.  Then:
  \begin{enumerate}
  \item If $k \le l$ and $\MS{loop}~k~f~h~s = \Some x$, then $\MS{loop}~l~f~h~s = \Some x$.
  \item If $\MS{loop}~k~f~h~s = \Some x$ and $\MS{loop}~l~f~h~s = \Some y$, then $x = y$.
  \item If $\MS{loop}~k~f~h~s = \Some{x}$, then $h(x) = \true$.
  \item If $h~s = \true$, then $\MS{loop}~l~f~h~s = \Some{s}$.
  \end{enumerate}
\end{lemma}
\begin{proof}
  Claims 1-3 follow by induction on $k:\Nat$.  Claim 4 follows by definition.
\end{proof}

For a machine $M$, we define the function $\MS{exec} \from \Tapes{n} \to \Nat \to \Option(\MS{Conf})$ that executes the $k$ steps of the machine,
given the initial tapes:
\begin{alignat*}{3}
  \MS{init\_state}   &~t               &&:= (t, start) \\
  \MS{halting\_state}&~(t, q)          &&:= halt(q) \\
  \MS{loopM}         &~\MS{initc}~k    &&:= \MS{loop}~k~\MS{step}~\MS{halting\_state}~\MS{initc} \\
  \MS{exec}          &~\MS{t}~k        &&:= \MS{loopM}~(\MS{init\_state}~t)~k
\end{alignat*}

We write $M(t) \terminates^k c$ for $\MS{exec}~t~k = \Some{c}$.


\section{Specification of Semantic}
\label{sec:spec_semantics}

We have defined a big-step semantic for mulit-tape Turing machines.  Now we want to define predicates to specify the semantic of a concrete
(parametrised) machine $M$.  The ``semantics'' of $M$ consists of two parts:  correctness and termination.

The correctness part is captured by \emph{realisation} of a (partitioned) relation $R$:

\begin{definition}[Realisation]
  \label{def:realisation}
  Let $M$ be a machine parametrised over $F$.  Let $R \subseteq \Tape_\Sigma^n \times F \times \Tape_\Sigma^n$.
  \[
    M \vDash R :=
    \forall t~k~q~t'.~M(t) \terminates^k (q, t') \rightarrow
    R~t~(\MS{part}~q)~t'
  \]
  Where $\MS{part} \from Q \to F$ is the partitioning function of $M$.
\end{definition}

If $M \vDash R$, we say that $M$ realises the relation $R$.  Informally, this means that the output of the machine is correct (i.e. is in the
relation), if the machine terminates.  The part implies termination of the machine on certain inputs, but also links the inputs to the number of steps
the machine needs for the computation.

\begin{definition}[Termination on a relation]
  \label{def:termination}
  Let $M$ be a (not partitioned) machine.  Let $T \subseteq \Tape_\Sigma^n \times \Nat$.
  \[
    M \downarrow T :=
    \forall t~k.~T~t~k \rightarrow
    \exists c.~M \terminates^k c.
  \]
\end{definition}


\begin{lemma}[Monotonicity of $M \vDash R$]
  \label{lem:Realise_monotone}
  If $M \vDash R$ and $R \subseteq R'$, then $M \vDash R'$.
\end{lemma}

\begin{lemma}[Monotonicity of $M \downarrow T$]
  \label{lem:TerminatesIn_monotone}
  If $M \downarrow T$ and $T' \subseteq T$, then $M \downarrow T'$.
\end{lemma}


For machines that always terminate in a constant number of steps, it is useful to combine both predicates:
\begin{definition}[Realisation in constant time]
  \label{def:RealiseIn}
  ~
  \[
    M \vDash^k R :=
    \forall t'.~
    \exists q~t'.~
    M(t) \terminates^k(q, t') \land R~t~(\MS{part}~q)~t'
  \]
\end{definition}

\begin{lemma}[Specification of realisation in constant time]
  \label{lem:Realise_total}
  \[
    M \vDash^k R
    \quad\iff\quad
    M \vDash R ~\land~
    M \downarrow (\lambda \_~i.~k \le i)
  \]
\end{lemma}



\section{Basic Machines}
\label{sec:basic_machines}

%%% Local Variables:
%%% TeX-master: "thesis"
%%% End:
