\chapter{Introduction}
\label{chap:intro}

Turing machines build the traditional foundation of the theory of computation and complexity.  However, reasoning about Turing machines can be
tedious, because their semantic is unstructured: from each state of the machine, the execution can continue in each other state, similar to the
infamous \textit{goto} statement \cite{dijkstra2002go}.

Textbooks like Boolos et~al.~\cite{boolos2007computability} leave out formal proofs of correctness.  To establish that a function is
Turing-computable, authors often give an informal description of the algorithm and conclude with the \textit{Church-Turing thesis} that the function
is Turing-computable.  Proofs of equivalence of Turing-computability and computability w.r.t.\ other abstract machine models are often informal.  They
leave out invariants and other rigorous details.  It can also happen that inconsistencies are discovered \cite{Xu:2013:MTM:2529315.2529331}.

Informal proofs are not enough for us.  We want proofs that are (1)~understandable by humans, and (2)~rigorous enough that even computers accept the
proof.  Formal proofs often involve a lot of little details that a human can easily oversee.  Computers can help in this regard.  They make it
possible to employ proof search and similar automation, to fill in \textit{tedious details}, so that the user can focus on the \textit{interesting}
parts of proofs.  Another vital aspect to ease the formalisation of complex but low-level abstract machines is the introduction of structure and
composability.  It requires a powerful framework, to define and certify complex machines with ease.  In this thesis, we introduce such a framework,
for programming and formally verifying multi-tape Turing machines.


\section{Contributions}
\label{sec:contributions}

We formalise a variant of deterministic multi-Tape Turing machines in the interactive theorem prover Coq~\cite{Coq}.  We build a framework for
programming and verifying Turing machines, that means giving a correctness, runtime, and termination certificate that is verified by Coq.  Our
framework extends the framework by Asperti and Ricciotti~\cite{asperti2015} in the interactive theorem prover Matita~\cite{asperti2011matita}.  We
make it easier to reason about Turing machines by eliminating the need to reason about concrete machine states, and introduce more general
control-flow operators.  We increase the level of programming abstraction and make it possible that machines can directly manipulate values of
arbitrary encodable types.  As a case-study, we show that our framework is strong enough to implement and verify a Turing machine that simulates a
two-stack machine.  We formally prove that the halting problem of this two-stack machine reduces to the halting problem of multi-tape Turing machines.

\section{Related Work}
\label{sec:relatedwork}

Asperti and Ricciotti~\cite{asperti2012} formalise single-tape Turing machines over arbitrary alphabets in the interactive theorem prover Matita.
Matita uses the same constructive type theoretic foundation as Coq, the calculus of (co)inductive constructions.  In~\cite{asperti2015}, they
formalise multi-tape Turing machines in Matita.  They use the notion of \textit{realisation} of relations for specifying semantics of Turing machines.
They define and verify a universal Turing machine and also formalise the reduction from multi-tape Turing machines to single-tape Turing machines.
Furthermore, they propose the formalisation of Turing machines as a benchmark for comparing proof assistants.

Xu, Zhang, and Urban~\cite{Xu:2013:MTM:2529315.2529331} formalise single-tape Turing machines over a binary alphabet in
Isabelle/HOL~\cite{nipkow2002isabelle}.  They follow the textbook of Boolos et~al.~\cite{boolos2007computability} and use Hoare-logic to specify the
semantics of Turing machines.  They translate \textit{abacus programs} and \textit{partial recursive functions} to Turing machines and prove the
undecidability of the halting problem of Turing machines.

Ciaffaglione et~al.~\cite{Ciaffaglione:2016:TTC:2956213.2956306} define tapes of Turing machines as infinite streams.  They certify Turing machines
using induction and co-induction in Coq.

In her bachelor thesis, Heiter~\cite{Heiter} reduces the halting problem of single-tape to the \textit{Post correspondence problem} in Coq.  They use
the same definition of single-tape Turing machines as in \cite{asperti2012}.



\section{Outline}
\label{sec:outline}

In Chapter~\ref{chap:definitions}, we define the notion of multi-tape Turing machines.  We also introduce means to specify the semantics of machines,
and define basic machines, on which all our machines are based.  In Chapter~\ref{chap:combining}, we control-flow operators.  In
Chapter~\ref{chap:lifting}, we show how to combine machines with different alphabets and numbers of tapes.  We combine simple machines in
Chapter~\ref{chap:compound}.  In Chapter~\ref{chap:programming}, we introduce abstractions that enable the programmer to directly manipulate values,
and show complex case-studies.  In Chapter~\ref{chap:heap} we develop our final case-study where we implement and verify a Turing machine that
simulates a two-stack machine and show that the halting problem of this machine reduces to the halting problem of multi-tape Turing machines.  We
conclude and discuss possible future work in Chapter~\ref{chap:conclusion}.  In the appendix, we present some pearls of the Coq development of this
thesis.  Note that in the PDF version of this thesis, all definitions and lemmas are hyperlinked to the documented online source code of the Coq
implementation of the framework.  It is also available for download, from:

\begin{center}
  \url{https://www.ps.uni-saarland.de/~wuttke/bachelor.php}
\end{center}


%%% Local Variables:
%%% TeX-master: "thesis"
%%% End:
