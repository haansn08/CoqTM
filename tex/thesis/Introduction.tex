\chapter{Introduction}
\label{chap:intro}

\section{Related Work}
\label{sec:relatedwork}

\section{Contributions}
\label{sec:contributions}

\section{Outline}
\label{sec:outline}

In Chapter \ref{chap:definitions}, the notion of multi-tape Turing machines is defined.
We also introduce our means to specify the semantics of machines, and define some basic machines.

Chapter \ref{chap:combining} defines control-flow operators and shows how we can combine machines.
We also show how to build some not-so-complex machine.

In Chapter \ref{chap:programming}, we introduce abstractions, like value-containing and computation of functions.
We introduce a general pattern, how to program multi-tape Turing machines; this pattern is followed in some case-studies.

In Chapter \ref{chap:heap}, we use all technics developed in this thesis, to build a machine that simulates an abstract heap machine.
From the correctness of this Turing machine, we conclude that the halting problem of the abstract machine reduces to the halting problem of multi-tape Turing machines.

We discuss some technical difficulties of our Coq-implementation in Chapter \ref{chap:implementation}.

Chapter \ref{chap:conclusion} concludes, and lists possibilities for future work.


%%% Local Variables:
%%% TeX-master: "thesis"
%%% End:
