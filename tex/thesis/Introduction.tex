\chapter{Introduction}
\label{chap:intro}


Although Turing machines are a simple (but not quite simplistic) model of computation, there are not many \textit{rigorous} proofs about Turing
machines in the literature.  We think that the following points are reasons for that.  First, their semantics is \textit{unstructured}: from each
state of the machine, the execution can proceed in every other state, similar to the infamous \textit{goto} statement~\cite{dijkstra2002go}.  But even
worse, Turing machines are not \textit{compositional}.  Sequential composition or loops of Turing machines are not \textit{per~se} available.  Even
the formal specification of machines is a burden, because complex machines may have a huge number of internal states.  Last but not least, they are
\textit{low-level}, because the operation on tapes are primitive: read a symbol from the tape, write a symbol to the tape, or move the (read/write)
head in a direction.

For the above reasons, textbooks like Boolos et~al.~\cite{boolos2007computability} leave out detailed proofs of correctness.  They also often only
give an informal description of machines, which obviously makes formal reasoning impossible.  Even if they define the whole machine, they leave out
formal specifications of invariants to figure out for the reader.  To establish that a function is Turing computable, authors often give an informal
description of the algorithm and conclude, using the \textit{Church-Turing thesis}, that the function is Turing-computable. Or they switch to another
abstract machine model, but define the compilation function between the models of computation only informally.

In this thesis, we aim to define, specify, and formally verify Turing machines in a framework built in the theorem prover Coq~\cite{Coq}.  First of
all, we address the problems above.  Instead of defining machines in terms of transition tables, we compose machines using functions of Coq's
dependent type theory -- the \textit{Calculus of (Co)Inductive Constructions} (also known as CIC).  For example, we define a function that builds the
sequential composition of two machines.  To eliminate the need to reason about concrete machine states, we give all states a label (e.g.\ $\true$ or
$\false$) and only have to reason about these labels.  The number of labels is always reasonable small, compared to the potential huge amount of
states.  We address the problem that machines are low-level, by introducing abstractions, so that we can define Turing machines that directly
manipulate values of arbitrary encodable types.  This gives the advantages of register machines, but we are not restricted to natural numbers.

There are many variants of Turing machines.  All variants can be shown to be computationally equivalent.  In this thesis, we choose multi-tape Turing
machines with arbitrary finite alphabets.  Our plan is that each tape should contain a value.  We choose a finite model of tapes.  This means that
each tape has only finitely (but arbitrarily) many symbols.


\section{Contributions}
\label{sec:contributions}

We formalise a variant of deterministic multi-tape Turing machines in the interactive theorem prover Coq.  We build a framework for programming and
formally verifying correctness and time complexity of Turing machines.  Our framework extends the framework by Asperti and
Ricciotti~\cite{asperti2015} in the interactive theorem prover Matita~\cite{asperti2011matita}.  Compared to their framework, we eliminate the need to
reason about concrete machine states and introduce more general control-flow operators.  We increase the level of programming abstraction and make it
possible that Turing machines can directly manipulate values of arbitrary encodable types.  We show that our framework is strong enough to implement
and verify a Turing machine that simulates a two-stack machine for a variant of the $\lambda$-calculus.  We formally prove that the halting problem of
this abstract machine reduces to the halting problem of multi-tape Turing machines.  Thereby, this work is the last step to formally prove that
multi-tape Turing machines can simulate the $\lambda$-calculus.

\section{Related Work}
\label{sec:relatedwork}

Asperti and Ricciotti~\cite{asperti2012} formalise single-tape Turing machines over arbitrary finite alphabets in the interactive theorem prover
Matita.  Matita uses the same constructive type theoretic foundation as Coq.  In~\cite{asperti2015}, they formalise multi-tape Turing machines in
Matita.  They introduce the notion of \textit{realisation} for specifying the semantics of concrete Turing machines.  They define and verify a
universal Turing machine and also formalise the reduction from multi-tape Turing machines to single-tape Turing machines.  Furthermore, they propose
the formalisation of Turing machines as a benchmark for comparing proof assistants.

Xu, Zhang, and Urban~\cite{xu2013} formalise single-tape Turing machines over a binary alphabet in Isabelle/HOL.  They follow the textbook of Boolos
et~al.~\cite{boolos2007computability} and use Hoare-logic to specify the semantics of concrete Turing machines.  They implement formally verified
translation functions from \textit{abacus programs} and \textit{partial recursive functions} to Turing machines and prove the undecidability of the
halting problem of Turing machines.

Ciaffaglione et~al.~\cite{ciaffaglione2016} define tapes of Turing machines as infinite streams.  They verify Turing machines using induction and
co-induction and also show the undecidability of the halting problem in Coq.

Forster, Heiter, and Smolka~\cite{forster2018verification} formally reduce the halting problem of single-tape Turing machines to the \textit{Post
  correspondence problem} (PCP) in Coq.  They use the same definition of Turing machines as we use, but restricted to one tape.  This definition of
single-tape Turing machines was originally presented in~\cite{asperti2012}.

There are other mechanisations of abstract machine models.  For example, Forster and Smolka~\cite{forster2017weak} formalise the theory of computation
in Coq, based on the language~$L$, which is also known as the (weak) call-by-value $\lambda$-calculus.  Norrish~\cite{norrish2011mechanised}
formalises computability theory in HOL4.  He considers a variant of the $\lambda$-calculus and recursive functions, and show that both models of
computation are computationally equivalent.  Kunze et~al.~\cite{KunzeEtAl:2018:Formal} formalise reductions from the programming language $L$ to
several stack-machines.  The stack-machine for that we build a simulator is a variant of a machine of this paper.



\section{Outline}
\label{sec:outline}

In Chapter~\ref{chap:definitions}, we define the notion of multi-tape Turing machines.  We also introduce means to specify the semantics of concrete
machines.  In Chapter~\ref{chap:basic}, we define primitive machines, on which all our machines are based.  In Chapter~\ref{chap:combining}, we define
control-flow operators.  In Chapter~\ref{chap:lifting}, we show how to combine machines with different alphabets and numbers of tapes.  We build
simple machines in Chapter~\ref{chap:compound}.  In Chapter~\ref{chap:programming}, we introduce abstractions that enable the programmer to directly
manipulate values, and we show complex case-studies.  In Chapter~\ref{chap:heap}, we develop our final case-study where we implement and verify a
Turing machine that simulates a two-stack machine for $L$ and show that the halting problem of this machine reduces to the halting problem of
multi-tape Turing machines.  We conclude and discuss possible future work in Chapter~\ref{chap:conclusion}.  In the appendix, we present pearls of the
Coq development of this thesis.

Throughout the thesis, we use mathematical notation, the reader is not required to be expert in type theory.  In the PDF version of this thesis, all
definitions and lemmas are hyperlinked to the documented online source code of the Coq implementation.  The source code is tested to compile with Coq
versions 8.7 and 8.8.  The home page of this thesis contains the PDF version, the source code, and online documentation:
\begin{center}
  \url\homepage
\end{center}

%%% Local Variables:
%%% TeX-master: "thesis"
%%% End:
