\chapter{Programming Turing Machines}
\label{chap:programming}

We can define machines using primitive operations like $\MS{DoAct}$, and can combine machines in an imperative programming style.  However, our
imperative ``language'' still has no notion of \emph{variables} or data cells.  We want to use each tape as a data cell to store one variable.
Therefore, we have to define what it means that a tape \emph{contains} a value.  When we ``\emph{program}'' Turing machines, instead of using basic
machines, we only use machines that directly change the value of a tape.  We use the combinators introduced in Chapter \ref{chap:combining} to
simulate control flow of imperative programming languages.  Using the definition of value-containing, we specify a ``callee-saving'' convention for
function computation.  We present a generic pattern how to program and verify Turing machines and show more complex case studies.


\section{Value-Containing}
\label{sec:value-containing}

We first want to define what it means that a tape $t$ \emph{contains} a value $x$, written as $t \simeq x$.  Tapes, as defined in \ref{def:tape}, are
essentially a list of symbols, so we have to linearise values to strings.

\begin{definition}[Encodable types]
  We say that a type $X$ is \emph{encodable} over $\Sigma$, if there is an injective function $encode : X \to \List(\Sigma)$.
\end{definition}

We write $encode_\Sigma$ if $\Sigma$ is not clear from the context.  For example, we can map encodings to different alphabets:

\begin{definition}[Map encodings]
  Let $X$ be encodable over $\Sigma$ and $f : \Sigma \to \Tau$ be injective.  Then $X$ is also encodable over $\Tau$ with the following encoding
  function:
  \[ encode_\Tau(x) = \map{f}{(encode_\Sigma(x))}. \]
\end{definition}


If $X$ is encodable over $\Sigma$, we encode values of $x$ on tapes with an extended alphabet $\Sigma^+$ that has an additional start symbol and end
symbol.

\begin{definition}[$\Sigma^+$] Let $\Sigma$ be an alphabet.
  \[
    \Sigma^+ ::= \MS{START} ~|~ \MS{STOP} ~|~ (s: \Sigma)
  \]
\end{definition}

Then we can define what $t \simeq x$ means:

\begin{definition}[$t \simeq x$]
  \label{def:tape_contains}
  Let $X$ be encodable over $\Sigma$ and $t : \Tape_{\Sigma^+}$.
  \[
    t \simeq x := \exists~ls.~
    t = \MS{midtape}~ls~(\MS{START})~(encode(x) \app [\MS{STOP}])
  \]
\end{definition}

Note that in Definition~\ref{def:tape_contains}, we have to map the encoding of $x$ to the extended alphabet $\Sigma^+$.  If $t \simeq x$, the head of
$t$ stands on the start symbol.  To the right, there is the encoding, which is terminated by the stop-symbol.  To the left of the head, there may be
arbitrary many ``rest'' symbols.

We found the convention usefull, that there are no further symbols beyond the stop symbol.  In future work we also want to reason about space usage of
machines.  If a tape contains a value, the size of the tape (i.e.\ the number of symbols on it), only depends on the length of the encoding and the
number of rest symbols $ls$ to the left.  We have the convention to only add symbols to the left of a tape.  We also only write on tapes, if the head
of the tape is on the right-most symbol.

\section{Value-Manipulating Machines}
\label{sec:value-manipulate}


We defined what it means that a tape contains a value.  Now we want to define machines that manipulate values, e.g\ increase or increase a number.
Also, a very useful operation is to copy a value from one tape to another tape.


\todo{}







%%% Local Variables:
%%% TeX-master: "thesis"
%%% End:
