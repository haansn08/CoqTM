\chapter{Programming Turing Machines}
\label{chap:programming}

We can define machines using primitive operations like $\MS{DoAct}$, and can combine machines in an imperative programming style.  However, our
imperative ``language'' still has no notion of \emph{variables} or data cells.  We want to use each tape as a data cell to store one variable.
Therefore, we have to define what it means that a tape \emph{contains} a value.  When we ``\emph{program}'' Turing machines, instead of using basic
machines, we only use machines that directly change the value of a tape.  We use the combinators introduced in Chapter \ref{chap:combining} to
simulate control flow of imperative programming languages.  Using the definition of value-containing, we specify a ``callee-saving'' convention for
function computation.  We present a generic pattern how to program and verify Turing machines and show more complex case studies.


\section{Value-Containing}
\label{sec:value-containing}

We first want to define what it means that a tape $t$ \emph{contains} a value $x$, written as $t \simeq x$.  Tapes, as defined in \ref{def:tape}, are
essentially a list of symbols, so we have to linearise values to strings.  We say that a type $X$ is \emph{encodable} over $\Sigma$, if we have
defined an injective function $encode_\Sigma : X \to \List(\Sigma)$.

If $X$ is encodable over $\Sigma$, we encode values of $x$ on tapes with an extended alphabet $\Sigma^+$.  $\Sigma^+$ has an additional start symbol
and end symbol.

\begin{definition}[$\Sigma^+$] Let $\Sigma$ be an alphabet.
  \[
    \Sigma^+ ::= \MS{START} ~|~ \MS{STOP} ~|~ \Sigma
  \]
\end{definition}

Then we can define what $t \simeq x$ means:

\begin{definition}[$t \simeq x$]
  \label{def:tape_contains}
  Let $X$ be encodable over $\Sigma$ and $t : \Tape_{\Sigma^+}$.
  \[
    t \simeq x := \exists~ls.~
    t = \MS{midtape}~ls~(\MS{START})~(encode(x) \app [\MS{STOP}])
  \]
\end{definition}

Note that in Definition~\ref{def:tape_contains}, we have to map the encoding of $x$ to the extended alphabet $\Sigma^+$.



%%% Local Variables:
%%% TeX-master: "thesis"
%%% End:
