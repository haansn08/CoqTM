\chapter{Lifting Machine}
\label{chap:lifting}

We observe that whenever we want to combine machines, the numbers of tapes and alphabets of all involved machines have to match.

For example, assume that we have a one-tape machine that moves the head to the right.  If we need a two-tape machine that moves both tapes to the
rigth, we would like to use sequencial composition and move one tape after the other tape to right.  But we would need two two-tape machines: one that
moves the first tape to the right and one that moves the second tape to the right.  There are multiple ways to solve this problem.  For example, we
could define a class of $n$-tape machines that move the $i$th tape to the right and leave all other tapes unchanged.  However, it is in general easier
to build and prove correctness of small machines that operate on a minimal amount of tapes.  Asperti and Ricciotti~\cite{asperti2015} defined an
\emph{injecting} operator that translates an one-tape machine into a $1+n$-tape machine.  We generalised this approach and implemented an operator
that takes an $m$-tape machine and yields an $n$-tape machine, where $m \le n$.

The second part of this problem is how to combine machines with different alphabets.  For example, if we have a machine $\MS{Add}$ that sums
(encodings of) natural numbers, we could want to build a machine $\MS{Sum}$ that sums numbers in a list of numbers.  Consider an alphabet
$\Sigma_\Nat$ where we could ``encode'' natural numbers on, and an alphabet $\Sigma_{\List(\Nat)}$ to ``encode'' lists of natural numbers.  If the
alphabet $\Sigma_\Nat$ is ``included'' in $\Sigma_{\List(\Nat)}$, we would like to ``lift'' $\MS{Add}$ to the alphabet $\Sigma_{\List(\Nat)}$, to
define $\MS{Sum}$.

Asperti and Ricciotti~\cite{asperti2015} avoid the latter problem.  They consider a fixed alphabet to ``encode'' all needed data on, to implement a
universal Turing machine.  However, this approach does not scale.  When the alphabet has to be changed, for example because we have to add a new
``data type'', they also must change the definitions of the involved ``smaller'' machines.  That is why we also introduce an operator that
\emph{lifts} a machine to a bigger alphabet.

\section{$n$-Lift}
\label{sec:n-Lift}

The $n$-Lift takes a machine $M:\TM_\Sigma^n(F)$.

\todo{}


\section{$\Sigma$-Lift}
\label{sec:sigma-Lift}


\todo{}




\section{Examples}
\label{sec:lifting-examples}

\todo{}


%%% Local Variables:
%%% TeX-master: "thesis"
%%% End: