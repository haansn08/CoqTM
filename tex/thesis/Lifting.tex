\chapter{Lifting Machines}
\label{chap:lifting}

We observe that whenever we want to combine machines, the number of tapes and alphabets of all sub-machines have to agree.  For example, we have the
following typing rule for the sequential composition of partitioned Turing machines:
\[
  \inferrule{M_1 : \TM_\Sigma^n(F_1) \and M_2 : \TM_\Sigma^n(F_2)}{M_1 \Seq M_2 : \TM_\Sigma^n(F_2)}
\]
For example, assume that we have a one-tape machine $M_{aux} : \TM_\Sigma^1$ that moves the head to the right of the tape.  If we need a two-tape
machine $M$ that moves both tapes to the right, we would like to use sequential composition and move one tape after the other tape to the right.  But
according to the typing rule above, we would need two auxiliary two-tape machines $M_1$ and $M_2$.  $M_1$ moves the first tape to the right and $M_2$
the second.

There are multiple ways to solve this problem.  Maybe the most obvious solution is to define a class of $n$-tape machines $M_i : \TM_\Sigma^n$
parametrised over the number of tapes and the tape-index $i$ of the tape to move.  The machine $M_i$ moves the $i$th tape to the right.  All other
tapes are ``inactive'' and remain unchanged.  This approach of parametrising machines over the tape-indexes of ``active'' tapes, however, becomes
unhandy for machines with a lot of active tapes.

We choose another approach.  We lift the one-tape machine $M_{aux}$ to $n$ different $n$-tape machines, for every $n\ge1$.  Asperti and
Ricciotti~\cite{asperti2015} implement such an operator that translates a one-tape machine into $n$-tape machines.  It is in general easier to define
machines with a fixed number of tapes.  We implement a generalised operator that takes an $m$-tape machine and a mapping from $m$-indexes to
$n$-indexes, and yields an $n$-tape machine, for~$m \le n$.

The second part of this problem is, how to combine machines with different alphabets.  For example, if we have a machine $\MS{Add}$ that sums
(encodings of) natural numbers, we could want to build a machine $\MS{Sum}$ that sums numbers in a list of numbers.  Consider an alphabet
$\Sigma_\Nat$ where we could encode natural numbers on, and an alphabet $\Sigma_{\List(\Nat)}$ to encode lists of natural numbers.  If the alphabet
$\Sigma_\Nat$ is included in $\Sigma_{\List(\Nat)}$, we would like to lift $\MS{Add}$ to the alphabet $\Sigma_{\List(\Nat)}$, to define $\MS{Sum}$.

Asperti and Ricciotti~\cite{asperti2015} avoid the latter problem: agreement of alphabets.  They consider a fixed alphabet to encode all needed data
on, to implement a universal Turing machine.  However, this approach does not scale when we need to encode many different data types.  Whenever the
alphabet has to be changed, they also must change the definitions of all auxiliary machines.  That is why we introduce another operator that lifts a
machine to a bigger alphabet.

Both lifting operators are easy to define and verify using the lemmas of the previous chapter (see Section~\ref{sec:match-proofs}).


\section{Tape-Lift}
\label{sec:n-Lift}
\setCoqFilename{ProgrammingTuringMachines.TM.Lifting.LiftTapes}%

The tape-lift takes a machine $M:\TM_\Sigma^m(F)$ and a duplicate-free vector $I: \Fin_n^m$, and yields a machine
$\LiftTapes{M}{I} : \TM_\Sigma^n(F)$.\footnote{ Note that instead of duplicate free index vectors $I:\Fin_n^m$, we could also use injective functions
  $f: \Fin_m \to \Fin_n$.  We choose index-vectors, because they are easier to define in Coq.}  The tape of $\LiftTapes{M}{I}$ with the index
$i = I[j]$ (with $i:\Fin_n$, $j:\Fin_m$) behaves exactly as the tape $j$ of $M$.  All other tapes of $\LiftTapes{M}{I}$ that are not in~$I$ are
inactive and do not change.


The transition function of $\LiftTapes{M}{I}$ gets the $n$ read symbols and selects the $m$ relevant symbols.  Then it applies the transition function
$\delta_M$ with the selected symbols and the current state $q$.  $\delta_M$ yields an $m$-vector $act:\Act^m$ and the continuation state $q'$.  It
fills ``nop''-actions into $act$, to get an action vector $act':\Act^n$.

\begin{definition}[Vector selecting][select]
  \label{def:select}
  Let $X:\Type$, $m,n:\Nat$, $I : \Fin_n^m$, and $V : X^m$.  $select~I~V : X^n$ with
  \[ select~I~V := \map{\bigl(\lambda (j:\Fin_m).~V[j] \bigr)}{I} \]
\end{definition}
\begin{lemma}[Correctness of $select$][select_nth]
  By definition, for $j:\Fin_m$, we have
  \[
    (select~I~V)[j]=V\bigl[I[j]\bigr]
  \]
\end{lemma}

\begin{definition}[Vector filling][fill]
  Let $X:\Type$, $m,n:\Nat$,$I:\Fin_n^m$, $init:X^n$, and $V:X^m$:
  \begin{alignat*}{3}
    & fill~(\nil      )&& ~init~V &~:=~& init \\
    & fill~(i \cons I')&& ~init~V &~:=~& replace~(fill~I'~(\tl V))~i~(\hd V)
  \end{alignat*}
  Where $replace : X^n \to \Fin_n \to X \to X^n$ replaces the $i$th element of a vector.
\end{definition}
\begin{lemma}[Correctness of $fill$]
  If $I$ is duplicate-free, then:
  \begin{enumerate}
  \coqitem[fill_correct_nth] \label{lem:fill_correct_nth}
    If $I[j]=i$, then $(fill~I~init~V)[i] = V[j]$.
  \coqitem[fill_not_index] \label{lem:fill_not_index}
    If $i \notin I$, then $(fill~I~init~V)[i] = init[i]$.
  \end{enumerate}
\end{lemma}
\begin{proof}
  By induction on $I : \Fin_n^m$.
\end{proof}

\begin{definition}[$\LiftTapes{M}{I}$][LiftTapes]
  \label{def:LiftTapes}
  Let $M:\TM_\Sigma^m(F)$ and $I : \Fin_n^m$.  We define $\LiftTapes{M}{I} : \TM_\Sigma^n(F)$.  All components are the same as in $M$, except:
  \begin{alignat*}{2}
    &\delta(q,s) &~:=~& \Let{(q', act) := \delta_M(q, select~I~sym)}{\\%
    &            &~  ~& (q', fill~I~(\None,N)^n~act)}
  \end{alignat*}
\end{definition}

\begin{lemma}[Correctness of $\LiftTapes{M}{I}$][LiftTapes_Realise]
  \label{lem:LiftTapes_Realise}
  Let $M : \TM_\Sigma^m(F)$ and $I : \Fin_n^m$ duplicate-free.  If $M \Realise R$, then $\LiftTapes{M}{I} \Realise \LiftTapes{R}{I}$ with
  \[
    \LiftTapes{R}{I} := \lambda t~(y,t').~ R~(select~I~t)~(y, select~I~t') \land \forall i:\Fin_n.~i \notin I \rightarrow t'[i]=t[i].
  \]
\end{lemma}

\begin{lemma}[Runtime of $\LiftTapes{M}{I}$][LiftTapes_TerminatesIn]
  \label{lem:LiftTapes_TerminatesIn}
  Let $M : \TM_\Sigma^m(F)$ and $I : \Fin_n^m$ duplicate-free. If $M \TerminatesIn T$, then $\LiftTapes{M}{R} \TerminatesIn \LiftTapes{T}{I}$ with
  $ \LiftTapes{T}{I} := \lambda t~k.~ T~(select~I~t)~k.  $
\end{lemma}

The proofs are similar to the former proofs, i.e.\ using Lemma~\ref{lem:loop_lift} and Lemma~\ref{lem:loop_unlift} with the following configuration
lifting function:
\[
  selectConf(q,t) := (q, select~I~t).
\]

However, for the second part of the correctness, i.e.\ tapes that are not in $I$ do not change, we need another lemma about $loop$:

\begin{lemma}[Mapping loops][loop_map]
  \label{lem:loop_map}
  Let $A:\Type$, $f:A \to A$, $h:A \to \Bool$.  Let $B:\Type$ and $g : A \to B$.  If $g(f~a)=g(a)$ for all $a:A$, then
  \begin{alignat*}{1}
    & \forall (k:\Nat)~(a_1~a_2 : A). \\
    & \quad loop~f~h~k~a_1 = \Some{a_2} \rightarrow \\
    & \quad g(a_1) = g(a_2).
  \end{alignat*}
\end{lemma}
\begin{proof}
  By induction on $k:\Nat$.
\end{proof}

We apply this lemma in the proof of Lemma~\ref{lem:LiftTapes_Realise} with $g := \lambda ((q,t):\Conf).~t[i]$.



\section{Alphabet-Lift}
\label{sec:sigma-Lift}
\setCoqFilename{ProgrammingTuringMachines.TM.Lifting.LiftAlphabet}%

Let $M : \TM_\Sigma^n(F)$ be a machine over the alphabet $\Sigma$, and $f : \Sigma \hookrightarrow \Tau$ a retraction on another alphabet $\Tau$.
Note that then $\Tau$ has at least as many symbols as $\Sigma$.  We need a default symbol $def:\Sigma$.  In contrast to the default partition we
needed in the definition of $\While$, the choice of $def$ is semantically relevant.  This means that $def$ should be a symbol that $M$ does not expect
to read.  The alphabet-lift $\LiftAlphabet{M}{(f,def)} : \TM_\Tau^n$ is a machine over the bigger alphabet $\Tau$.

The transition function of the lifted machine $\LiftAlphabet{M}{(f,def)}$ reads the $n$ optional symbols $s : \Option(\Tau)^n$.  It tries to translate
these symbols to $s': \Option(\Sigma)^n$, using the partial inversion function $f^{-1} : \Tau \to \Option(\Sigma)$.  If the symbol $\tau:\Tau$ has no
corresponding symbol in $\Sigma$, it must be translated to $def$ instead.  The transition function $\delta_M$ of $M$ yields the successor state $q'$
and a vector of actions $act: (\Option(\Sigma) \times \MS{Move})^n$, which is translated using $f:\Sigma\to\Tau$ to
$act' : (\Option(\Tau) \times \MS{Move})^n$.

\begin{definition}[$\LiftAlphabet{M}{(f,def)}$][LiftAlphabet]
  \label{def:LiftAlphabet}
  Let $f : \Sigma \hookrightarrow \Tau$, $def:\Sigma$, and $M : \TM_\Sigma^n(F)$.  We define the machine %
  $\LiftAlphabet{M}{(f,def)} : \TM_\Tau^n(F)$ with the same components as $M$, except:
  \begin{alignat*}{2}
    & \delta(q,s)    &~:=~& \Let{(q', act) := \delta_M(q, \map{(mapOpt~(surject~f~def))}{s})}{\\%
    &                &~  ~& (q', \map{(mapAct~f)}{act})} \\
  \end{alignat*}
  With $surject~f~def : \Tau \to \Sigma$:
  \[
    \coqlink[surjectTape]{surject}~f~def~\tau :=
    \begin{cases}
      \sigma & f^{-1}(\tau) = \Some{\sigma} \\
      def & f^{-1}(\tau) = \None
    \end{cases}
  \]
  and with the canonical functions $mapOpt: \forall (X~Y:\Type).~ (X \to Y) \to \Option(X) \to \Option(Y)$ and
  $mapAct : \forall (\Sigma~\Tau:\Type).~(\Sigma\to\Tau) \to \Act_\Sigma \to \Act_\Tau$.
\end{definition}

\setCoqFilename{ProgrammingTuringMachines.TM.TM}%
For the correctness and running time lemmas, we also need the canonical function
\[
  \coqlink[mapTape]{mapTape}:\forall (\Tau~\Sigma:\Type).~(\Tau\to\Sigma)\to\Tape_\Tau\to\Tape_\Sigma
\]
that maps every symbol on a tape.  We write $\coqlink[mapTapes]{mapTapes}$ for the respective tape-vector function.
\setCoqFilename{ProgrammingTuringMachines.TM.Lifting.LiftAlphabet}%

\begin{lemma}[Correctness of $\LiftAlphabet{M}{(f,def)}$][LiftAlphabet_Realise]
  \label{lem:LiftAlphabet_Realise}
  If $M \Realise R$, then $\LiftAlphabet{M}{(f,def)} \Realise \LiftAlphabet{R}{(f,def)}$ with
  \small
  \[
    \LiftAlphabet{R}{(f,def)} := \lambda t~(y, t').~R~(mapTapes~(surject~f~def)~t)~(y, mapTapes~(surject~f~def)~t').
  \]
\end{lemma}

\begin{lemma}[Runtime of $\LiftAlphabet{M}{(f,def)}$][LiftAlphabet_TerminatesIn]
  \label{lem:LiftAlphabet_TerminatesIn}
  If $M \TerminatesIn T$, then $\LiftAlphabet{M}{(f,def)} \TerminatesIn \LiftAlphabet{T}{(f,def)}$ with
  \[
    \LiftAlphabet{T}{(f,def)} := \lambda t~k.~T~(mapTapes~(surject~f~def)~t)~k.
  \]
\end{lemma}

The proofs are analogous to the former proofs.  The configuration lifting is
\[
  \coqlink[surjectConf]{surjectConf} (q,t) := (q, mapTapes~(surject~f~def)~t).
\]



%%% Local Variables:
%%% TeX-master: "thesis"
%%% End: