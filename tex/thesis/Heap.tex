\chapter{Simulating a Heap Machine}
\label{chap:heap}

\newcommand{\subst}[3]{{#1}^{#2}_{#3}}
\newcommand{\Ter}{\MS{Ter}}
\newcommand{\red}{\succ}


In this chapter, we present a large benchmark for our framework.  We implement a multi-tape Turing machine that simulates another abstract machine.
First, we define and motivate the semantics of this machine.  After that, we implement and verify the simulator Turing machine.  We show that the
halting problem of the abstract machine reduces to the halting problem of multi-tape Turing machines.

\section{Heap Machine}
\label{sec:heap}

The abstract machine we present here is a variant of the heap machine in Kunze et~al.~\cite{KunzeEtAl:2018:Formal}.  They formally show in Coq, that
their variant of the machine can simulate the language $L$, which implements the ``call-by-value'' subset of the $\lambda$-calculus.  Thus, we
implement a simulator that also simulates $L$.  Here we define the language $L$ only briefly.  For more a more thoughtful treatment of this language,
we refer to~\cite{forster2017weak} and~\cite{KunzeEtAl:2018:Formal}.


\subsubsection{Call-By-Value $\lambda$-Calculus}
\label{sec:L}

Terms of this language are De~Bruijn terms, and are inductively defined by:
\begin{align*}
  s,t,u,v &~:~\Ter~::=~ (n:\Nat) ~|~ s~t ~|~ \lambda s
\end{align*}
The language $L$ uses simple substitution:
\begin{align*}
  \subst kku &~:=~u \\
  \subst nku &~:=~n &&\text{if}~n\neq k \\
  \subst{(st)}ku &~:=~(\subst sku)(\subst tku) \\
  \subst{(\lambda s)}ku &~:=~\lambda(\subst s{Sk}u)
\end{align*}
The reduction relation $s \red t$ is defined inductively on terms:
\begin{mathpar}
  \inferrule*{~}{(\lambda s) (\lambda t)\red\subst s0{\lambda t}} \and%
  \inferrule*{s\red s'} {st\red s't} \and%
  \inferrule* {t\red t'} {(\lambda s)t\red(\lambda s)t'}%
\end{mathpar}


\subsubsection{Heap Machine}
\label{sec:heap-def}
\setCoqFilename{ProgrammingTuringMachines.TM.LM.Semantics}%

Instead of substituting expressions, the heap machine works with closures.  A closure is a program together with an environment.  Variable bindings
are implemented as pointers to a linked list of closures, called \textit{heap}.  Programs are lists of commands:
\begin{definition}[Program][Pro]
  \label{def:Com}
  \begin{alignat*}{2}
    n:   &~Var  &~:=~& \Nat \\
    c:   &~Com  &~::=~& \MS{VAR}(n:Var) ~|~ \MS{APP} ~|~ \MS{LAM} ~|~ \MS{RET} \\
    P,Q: &~Pro  &~:=~& \List(Com)
  \end{alignat*}
\end{definition}

Programs essentially are linearised expressions of $L$.  The function $\gamma$ translates terms of $L$ to programs of the heap machine:
\begin{alignat*}{2}
  &\gamma&~n           &:= [\MS{VAR}~n] \\
  &\gamma&~(s~t)       &:= \gamma~s \app \gamma~t \app [\MS{APP}] \\
  &\gamma&~(\lambda s) &:= \MS{LAM} \cons \gamma~s \app [\MS{RET}]
\end{alignat*}

The heap is implemented as a list of heap entries.  A heap entry may be empty or contain a closure and a pointer to the next heap entry.  Pointers are
implemented as list indices on the heap entry list.
\begin{definition}[Closures and heaps][Heap]
  \label{def:Heap}
  ~
  \begin{alignat*}{2}
    a,b: &~Add  &~:=~& \Nat \\
    g:   &~Clos &~:=~& Add \times Pro \\
    e:   &~Entr &~:=~& \Option(Clos \times Add) \\
    H:   &~Heap &~:=~& \List(Entr)
  \end{alignat*}
\end{definition}

States of the abstract machine are triples of two closure lists and a heap:
\[
  (T, V, H) ~:~ \List(Clos) \times \List(Clos) \times Heap.
\]
$T$ is called the control stack.  It contains the closures that the machine has to process.  The second stack $V$ is called the argument stack.  The
$\MS{APP}$ command fetches the functions and arguments from that stack.


We first explain what each rule of the reduction predicate $(T,V,H) \succ (T',V',H')$ does, and then we define it formally.


\paragraph{Application Rule}
When the first closure on the control stack is $(a, \MS{APP} \cons P)$, the machine fetches two closures $g$ and $(b,Q)$ from the argument stack.  The
closure $g$ corresponds to the argument of the application.  The second closure $(b,Q)$ corresponds to the called function, where $b$ is the pointer
to the environment in that the argument is free.  It binds the argument to $g$, by putting a new heap entry $(g,b)$ on the heap.  The heap machine
continues executing $(c,Q)$ and the ``rest closure'' $(a,P)$, where $c$ is the pointer to the new heap entry.


\paragraph{$\lambda$-Rule}
If the head control closure is $(a, \MS{LAM} \cons P)$, the machine splits the linearised program $P$ into body $Q$ of the $\lambda$-expression and
the rest program $P'$.  Then it pushes the closure of the rest program $(a,P')$ on the control stack, and the ``body closure'' $(a,Q)$ on the argument
stack.  The splitting is realised with the function $\phi : Pro \to \Option(Pro \times Pro)$.  For example, {\small
  \[
    \phi[\MS{VAR}(0); \MS{LAM}; \MS{APP}; \MS{RET}; \MS{RET}; \MS{VAR}(1)] = \Some{([\MS{VAR}(0); \MS{LAM}; \MS{APP}; \MS{RET}], [\MS{VAR}(1)])}.
  \]
}%
The first $\MS{RET}$ matches to the $\MS{LAM}$ in the program above, so it is part of the first partition.  $\phi$ is formally defined with a
tail-recursive auxiliary function:
\begin{definition}[$\phi$][jumpTarget]
  \label{def:jumpTarget}
  We define $\phi~P :=~ \phi'~0~\nil~P$ with the auxiliary function $\phi' : \Nat \to Pro \to Pro \to \Option(Pro \times Pro)$, which is defined by
  recursion on $P:Pro$.
  \begin{alignat*}{4}
    &\phi'&~0~   &~Q&~(\MS{RET} \cons P) &&~:=~& \Some{(Q,P')} \\
    &\phi'&~(S~k)&~Q&~(\MS{RET} \cons P) &&~:=~& \phi'~k~(Q \app [\MS{RET}])~P \\
    &\phi'&~k    &~Q&~(\MS{LAM} \cons P) &&~:=~& \phi'~(S~k)~(Q \app [\MS{LAM}])~P \\
    &\phi'&~k    &~Q&~(c        \cons P) &&~:=~& \phi'~k~(Q \app [c])~P \qquad \text{if $c \neq \MS{RET} \neq \MS{LAM}$} \\
    &\phi'&~k    &~Q&~              \nil &&~:=~& \None
  \end{alignat*}
\end{definition}


\paragraph{Variable Rule}
If the first command on the first closure of the control stack is $\MS{VAR}(n)$, the machine looks up the $n$th entry in the environment on the heap
at the address $a$.  Then, it pushes this closure to the argument stack.  The function $lookup : Heap \to Add \to Var \to \Option(Clos)$ starts at the
heap entry at the address $a$, and gets the $n$th entry. We write $H[a,n]$ for $lookup~H~a~n$.  The function is defined by recursion on~$n$:
\begin{definition}[$lookup$][lookup]
  \label{def:lookup}
  $H[a,n]$ is defined by recursion on $n$:
  \[
    H[a,n] :=
    \begin{cases}
      \Some{g} & H[a]=\Some{\Some{(g,b)}} \land n=0 \\
      H[b,n-1] & H[a]=\Some{\Some{(g,b)}} \land n>0 \\
      \None & \text{else}
    \end{cases}
  \]
  where $H[\cdot] : \Option(Entr)$ is the standard list lookup function.
\end{definition}


We formally define the reduction rules.  Note that there is no reduction rule for the command $\MS{RET}$, because the purpose of $\MS{RET}$ is solely
to mark the end of the encoding of the body of a $\lambda$-expression.
\begin{definition}[Semantics of the heap machine]
  {\small
    \begin{alignat*}{4}
      & \left((a, (\MS{APP} \cons P)) \cons T, g \cons (b, Q) \cons V, H\right) &~\red~& \left((c, Q) \cons (a,P) \cons_{tr} T, V, H'\right)
      \quad&& \text{if $put~H~\Some{(g,b)} = (c,H')$} \\
      & \left((a, (\MS{LAM} \cons P)) \cons T, V, H\right)                      &~\red~& \left((a, P') \cons_{tr} T, (a,Q) \cons V, H\right)
      \quad&& \text{if $\phi~P = \Some{(Q, P')}$} & \\
      & \left((a, (\MS{VAR}(n) \cons P)) \cons T, V, H\right)                   &~\red~& \left((a, P) \cons_{tr} T, g \cons V, H\right)
      \quad&& \text{if $H[a,n]=\Some{g}$} &
    \end{alignat*}
  }
  with $put~H~c := (\length{H}, H \app [c]) $.\\
  The tail-recursion optimisation $(g \cons_{tr} T) : \List(Clos)$ is defined as follows:
  \begin{alignat*}{3}
    & (a, \nil) &~\cons_{tr}~& T &~:=~& T \\
    & (a, P   ) &~\cons_{tr}~& T &~:=~& (a, P) \cons T \qquad\text{if $P \neq \nil$}
  \end{alignat*}

  We write for $\red^*$ for the reflexive transitive closure of $\red$ (cf.\ Definition~\ref{def:Kleene}) and $\red^k$ for the relational power (cf.\
  Definition~\ref{def:pow}).  Furthermore, we use the following notations:
  \begin{align*}
                         halt(T,V,H) &:= \forall T'~V'~H'.~\lnot \left((T,V,H) \red (T',V',H')\right) \\
    (T,V,H) \terminates^k (T',V',H') &:= (T,V,H) \red^k (T',V',H') \land halt(T',V',H') \\
    (T,V,H) \terminates^* (T',V',H') &:= (T,V,H) \red^* (T',V',H') \land halt(T',V',H')
  \end{align*}
\end{definition}

The tail-recursion optimisation ($\cons_{tr}$) makes sure that no closures with empty programs are pushed on a closure stack.  This optimises space
usage for tail recursive programs.

\begin{lemma}[Basic properties about $\red$]
  \label{lem:heap-red}
  ~
  \begin{enumerate}
    \coqitem[step_iff] \label{lem:step-iff}%
    The relation $\red$ is functional and computable, i.e.\ there is a function
    $step : \List(Clos) \times \List(Clos) \times Heap \to \Option\left(\List(Clos) \times \List(Clos) \times Heap\right)$, such that:
    \[
      step(T,V,H)=\Some{(T',V',H')} ~\iff~ (T,V,H) \red (T',V',H')
    \]
    \coqitem[is_halt_state_correct] \label{lem:halt_state_dec}%
    $halt(T,V,H)$ is decidable, i.e.\ there is a function $isHalt : \List(Clos) \times \List(Clos) \times Heap \to \Bool$, such that:
    \[
      isHalt(T,V,H) = \true ~\iff~ halt(T,V,H)
    \]
  \end{enumerate}
\end{lemma}


\section{Implementation of a Simulator}
\label{sec:heap-implementation}

First, we notice that all types in Definition~\ref{def:Heap}, except commands ($Com$), are encodable over canonical alphabets, according to
Definition~\ref{def:basic-encodings}.  \setCoqFilename{ProgrammingTuringMachines.TM.LM.Alphabets}%
\begin{definition}[Encoding of heaps][]
  \label{def:Heap-encode}
  The type $Com$ is isomorphic to the sum of $ACom := \MS{APP} ~|~ \MS{LAM} ~|~ \MS{RET}$ and $Var$.  $ACom$ is encodable on itself (since it is
  finite), and $Var = \Nat$ is encodable on $\Sigma_{Var} := \Sigma_\Nat$.  Then $Com$ is encodable on $\Sigma_{Com} := \Sigma_{ACom+\Nat}$.  All
  other types are encodable according to Definition~\ref{def:basic-encodings}.
  For completeness, the alphabets are:\\
  $\Sigma_{Pro} := \Sigma_{\List(Com)}$, $\Sigma_{Add} := \Sigma_\Nat$, $\Sigma_{Clos}:=\Sigma_{Add \times Pro}$,
  $\Sigma_{Entr} := \Sigma_{\Option(Clos \times Add)}$, and $\Sigma_{Heap} := \Sigma_{\List(Entr)}$.
\end{definition}

First, we derive constructor and deconstructor machines for commands.  As the type of commands essentially is the sum of two encodable types ($ACom$
and $\Nat$), we can combine constructors and deconstructors from Section~\ref{sec:constructors-deconstructors}.
\setCoqFilename{ProgrammingTuringMachines.TM.LM.MatchTok}%
\begin{definition}[$\MS{MatchCom}$][MatchTok]
  $\MS{MatchCom} : \TM_{\Sigma^+_{Com}}^1(\Option(ACom))$ is defined by
  \begin{alignat*}{2}
    & \MS{MatchCom} &~:=~& \MS{If}~\MS{MatchSum} \\
    &               &    & \MS{Then}~   \Return{\Nop}{\None} \\
    &               &    & \MS{Else}~~~ \MS{ChangePartition}~(\LiftAlphabet{\MS{MatchFin}}{RetrRgt})~(\Some\cdot)
  \end{alignat*}
\end{definition}
Note that $RetrRgt : \Sigma_{Var} \hookrightarrow \Sigma_{Com}$.
\begin{lemma}[Correctness of $\MS{MatchCom}$][MatchTok_Sem]
  \label{lem:MatchCom_Sem}
  $\MS{MatchCom} \RealiseIn{11} MatchCom$ with
\begin{lstlisting}[style=semicoqstyle]
$MatchComRel :=$
  $\lambda t~(y, t').~\forall (c:Com).~ t[0] \simeq c \rightarrow \MS{match}~y,c$
      [$\Some{\MS{APP}}, \MS{APP} \Rightarrow isRight~t'[0]$
      |$\Some{\MS{LAM}}, \MS{LAM} \Rightarrow isRight~t'[0]$
      |$\Some{\MS{RET}}, \MS{RET} \Rightarrow isRight~t'[0]$
      |$\None, \MS{VAR}~n \Rightarrow t'[0] \simeq n$
      |_,_=>$\False$
      ].
\end{lstlisting}
\end{lemma}
The $\MS{VAR}$ constructor is defined with $\MS{ConstrInr}$ and the constructor for $ACom$ is defined with $\MS{WriteSymbol}$.

For the functions $\phi$ and $lookup$, we define machines that \textit{compute} these functions in a lesser strict sense than
Definition~\ref{def:Computes2_Rel}.  The Turing machines only save the input-values that a caller machine needs again.  For example, we have a machine
$\MS{Nth'}$ that saves the list, but not the index.  The second change is that for functions with optional outputs, like $nth$, we do not extend the
alphabet with $\Sigma_{\Option(X)}$.  Instead, we encode whether the output is $\Some{\cdot}$ or $\None$ with the partition $\Bool$.  In case that the
output is $\None$, the machine terminates in $\false$ and the output tape stays right.

After we have defined and verified machines that compute $\phi$ and $lookup$, we define a machine $\MS{Step}$ that simulates a single step of the heap
machine.


\subsection{$\MS{Lookup}$}
\label{sec:Lookup}
\setCoqFilename{ProgrammingTuringMachines.TM.LM.LookupTM}%

The machine $\MS{Lookup} : \TM_{\Sigma^+}^5(\Bool)$ realises the heap $lookup$ function.  Note, that the heap is just a linked list of closures, where
the pointers are indices.  It uses the machine $\MS{Nth}' : \TM_{\Sigma^+}^4(\Bool)$ which is like $\MS{Nth}$ from Section~\ref{sec:Nth}, but with the
two changes mentioned above.

First we consider the alphabet of $\MS{Lookup}$.  We can encode closures on the heap alphabet $\Sigma_{Heap}$.  However when $\MS{Step}$ calls
$H[a,n]$, $a$ and $n$ are encoded on an alphabet of the closures of a closure stack.  Thus, we define $\MS{Lookup}$ on an alphabet $\Sigma$, with
retractions $f_{Heap} : \Sigma_{Heap} \hookrightarrow \Sigma$ and $f_{Clos} : \Sigma_{Clos} \hookrightarrow \Sigma$.  The second retraction correspond
to the closure alphabet where $a$ and $n$ are encoded.  The retraction
\[
  \coqlink[retr_clos_lookup_heap]{f_{Clos'}} : \Sigma_{Clos} \hookrightarrow \Sigma := RetrFst \circ RetrOpt \circ RetrList \circ f_{Heap}
\]
is the canonical retraction of closures from the heap.

There are three relevant retractions $\Sigma_\Nat \hookrightarrow \Sigma$, depending on whether the number is:
\begin{enumerate}
\item a heap address of an closure on the stack alphabet:
  \[
    \coqlink[retr_nat_lookup_clos_ad]{f_{add}} := RetrFst \circ f_{Clos'}
  \]
\item a variable in a command of a closure on the stack alphabet:
  \[
    \coqlink[retr_nat_lookup_clos_var]{f_{var}} := RetrRgt \circ RetrList \circ RetrSnd \circ f_{Clos'}
  \]
\item a pointer to the next heap entry:
  \[
    \coqlink[retr_nat_lookup_entry]{f_{next}} := RetrSnd \circ RetrOpt \circ RetrList \circ f_{Heap}
  \]
\end{enumerate}

Since the function $lookup$ is tail-recursive, we define the step machine with the type $LookupStep : \TM_{\Sigma^+}^5(\Option(\Bool))$.  It first
calls the list lookup machine ($\MS{Nth'}$).  If it failed, $\MS{Lookup}$ immediately terminates in $\false$.  Else, it makes a case-distinction on
the heap entry $e : \Option(Clos \times Add)$.  In case it is $\None$, $\MS{Lookup}$ also terminates in $\false$.  Else it destructs the closure
$g = (P,b)$ and makes a case-distinction over $n$.  If $n=0$, $\MS{LookupStep}$ resets the tapes for $b$ and $n$, translates $g$ from $f_{Clos'}$ to
$f_{Clos}$ and returns $\true$.  Else it translates $b$ from $f_{next}$ to $f_{add}$, moves it to the tape that contained $a$, and repeats the loop.
We visualise the execution in the latter two cases in Tables~\ref{tab:exec-LookupStep-S} and~\ref{tab:exec-LookupStep-0}.

\begin{table}[H]
  \scriptsize
  \begin{tabular}{l||l|l|l|l|l|l|l|l}
    Input       & $\MS{Nth'}$      & $\MS{MatchOpt}$       & $\MS{MatchPair}$     & $\MS{MatchNat}$ & $\MS{CopyValue}$  & $\MS{Translate}$ & $\MS{Reset}$ & $\MS{Reset}$ \\ \hline
    $0:H$       & $0:H$            &                       &                      &                 &                   &                  &              &              \\
    $1:a$       & $1:~\dashv$      &                       &                      &                 & $1:b$             & $0:b$            &              &              \\
    $2:n$       &                  &                       &                      & $0:n'$          &                   &                  &              &              \\
    $3:~\dashv$ &                  &                       & $1: g$               &                 &                   &                  &              & $0:~\dashv$  \\
    $4:~\dashv$ & $2:\Some{(g,b)}$ & $0:(g,b)$             & $0:b$                &                 & $0:b$             &                  & $0:~\dashv$  &              \\
  \end{tabular}
  \caption{Execution protocol of $\MS{LookupStep}$ in case $H[a] = \Some{\Some{(g,b)}}$ and $n=S~n'$.  It terminates in the partition $\None$.  The
    translation is from $f_{next}$ to $f_{add}$.}
  \label{tab:exec-LookupStep-S}
  \begin{tabular}{l||l|l|l|l|l|l|l}
    Input       & $\MS{Nth'}$      & $\MS{MatchOpt}$       & $\MS{MatchPair}$     & $\MS{MatchNat}$ & $\MS{Reset}$ & $\MS{Reset}$ & $\MS{Translate}$ \\ \hline
    $0:H$       & $0:H$            &                       &                      &                 &              &              &                  \\
    $1:a$       & $1:~\dashv$      &                       &                      &                 &              &              &                  \\
    $2:n$       &                  &                       &                      & $0:0$           &              & $0:~\dashv$  &                  \\
    $3:~\dashv$ &                  &                       & $1: g$               &                 &              &              & $0:g$            \\
    $4:~\dashv$ & $2:\Some{(g,b)}$ & $0:(g,b)$             & $0:b$                &                 & $0:~\dashv$  &              &                  \\
  \end{tabular}
  \caption{Execution protocol of $\MS{LookupStep}$ in case $H[a] = \Some{\Some{(g,b)}}$ and $n=0$.  It terminates in the partition $Some\true$.  The
    translation is from $f_{Clos'}$ to $f_{Clos}$.}
  \label{tab:exec-LookupStep-0}
\end{table}

\begin{definition}[$\MS{LookupStep}$][Lookup_Step]
  \label{def:Lookup_Step}
  We define the machine $\MS{LookupStep} : \TM_{\Sigma^+}^5(\Option(\Bool))$.
  \begin{align*}
    & \MS{LookupStep} := \\
    & \quad \MS{If}~(\MS{Nth'}~f_{heap}~f_{add}) \\
    & \quad \MS{Then}~\MS{If}~(\LiftBoth{\MS{MatchOption}}{RetrList ~\circ~ f_{heap}}{\Vector{4}}) \\
    & \quad \phantom{\MS{Then}}~\MS{Then}~\LiftBoth{\MS{MatchPair}}{RetrOpt \circ RetrList ~\circ~ f_{heap}}{\Vector{4;3}} \Seq \\
    & \quad \phantom{\MS{Then}~\MS{Then}}~\MS{If}~(\LiftBoth{\MS{MatchNat}}{f_{var}}{\Vector{2}}) \\
    & \quad \phantom{\MS{Then}~\MS{Then}}~\MS{Then}~ \MS{Return}~\bigl(
      \LiftTapes{\MS{CopyValue}}{\Vector{4;1}} \Seq
      \LiftTapes{\MS{Translate}~f_{next}~f_{add}}{\Vector{1}} \Seq \\
    & \quad \phantom{\MS{Then}~\MS{Then}~\MS{Then}~\MS{Return}\bigl(}
      \LiftTapes{\MS{Reset}}{\Vector{4}} \Seq
      \LiftTapes{\MS{Reset}}{\Vector{3}}
      \bigr)~{\None} \\
    & \quad \phantom{\MS{Then}~\MS{Then}}~\MS{Else}~~~ \MS{Return}~\bigl(
      \LiftTapes{\MS{Reset}}{\Vector{4}} \Seq
      \LiftTapes{\MS{Reset}}{\Vector{2}} \Seq \\
    & \quad \phantom{\phantom{\MS{Then}~\MS{Then}}~\MS{Else}~~~\MS{Return}~\bigl(}
      \LiftTapes{\MS{Translate}~f_{Clos'}~f_{Clos}}{\Vector{3}}
      \bigr)~{\Some\true} \\
    & \quad \phantom{\MS{Then}}~\MS{Else}~~~ \Return{\Nop}{\Some\false} \\
    & \quad \MS{Else}~~~ \Return{\Nop}{\Some\false}
  \end{align*}
\end{definition}

The step machine terminates in $\Some\false$ in two cases.  Either the list lookup failed, or it returned the empty heap entry $H[a]=\Some\None$.  In
the successfully termination case (cf.\ Table~\ref{tab:exec-LookupStep-0}), $\MS{LookupStep}$ resets the afterwards unneeded tapes.

Because no initialisation or clean-up is needed before or after the loop, we can define $\MS{Lookup} := \While~\MS{LookupStep}$.  The correctness
relation of $\MS{Lookup}$ says that if it terminated in the partition $\true$, we have $H[a,n] = \Some{g}$ for some closure $g$, and tape~$3$ contains
$g$ w.r.t.\ the retraction $f_{Clos}$.  Furthermore, the input tape~$0$ still contains $H$ and all other tapes are reseted.  In case $\MS{Lookup}$
terminated in $\false$, the postcondition only asserts that $H[a,n]=\None$.  Note that there are two possible reasons for that: Either $H[a]=\None$ or
$H[a]=\Some\None$, but we do not separate these two failure cases.

\begin{lemma}[Correctness of $\MS{Lookup}$][Lookup_Step_Realise]
  \label{lem:Lookup_Realise}
  $\MS{Lookup} \Realise LookupRel$
  with
  \small
  \begin{alignat*}{1}
    & LookupRel := \\
    &\quad \lambda t~(y, t').~\forall H~a~n.~ t[0] \simeq H \rightarrow t[1] \simeq_{f_{add}} a \rightarrow t[2] \simeq_{f_{var}} n \rightarrow isRight~t[3] \rightarrow isRight~t[4] \rightarrow \\
    &\quad\quad \MS{if}~y~\MS{then}~\bigl(\exists g.~H[a,n] = \Some{g} \land \\
    &\quad\quad \phantom{\MS{if}~y~\MS{then}~\bigl(\exists g.~} t'[0] \simeq H \land isRight~t'[1] \land isRight~t'[2] \land t'[3] \simeq_{f_{Clos}} g \land isRight~t[4]\bigr) \\
    &\quad\quad \MS{else}~ H[a,n] = \None.
\end{alignat*}
\end{lemma}


\subsection{$\MS{SplitBody}$}
\label{sec:SplitBody}
\setCoqFilename{ProgrammingTuringMachines.TM.LM.JumpTargetTM}%

The machine $\MS{SplitBody} : \TM_{\Sigma^+_{Pro}}^5(\Bool)$ computes the function $\phi$ (cf.\ Definition~\ref{def:jumpTarget}).  The
implementation of this machine is straight-forward, because $\phi'$ is defined tail-recursively.  The machine
$\MS{SplitBodyLoop} := \While~\MS{SplitBodyStep}$ computes $\phi'$, and $\MS{SplitBody}$ initialises the accumulators before executing the loop.

The step machine $\MS{SplitBodyStep} : \TM_{\Sigma^+_{Pro}}^5(\Option(\Bool))$ first makes a case-distinction over the program on tape $0$.  In the
$\nil$ case, it returns $\Some{\false}$ so that the loop returns $\false$.  In the cons-case, it makes a case-distinction over the command, using
$\MS{MatchCom}$.  In the case the head command is $\MS{RET}$, it does another case-distinction over $k$.  In the case that the head command is
$\MS{VAR}(n)$, the machine again applies the $\MS{VAR}$ constructor to $n$ and appends this command to $Q$.  The case-machines use the auxiliary
parametrised machine $\MS{AppACom}~t$ that appends a finite command (i.e.\ $t$ is either $\MS{APP}$, $\MS{LAM}$, or $\MS{RET}$) to the command list,
and the machine $\MS{AppCom}$ that appends a command on a tape to $Q$.


\begin{definition}[$\MS{SplitBody}$][JumpTarget_Step]
  \label{def:JumpTarget}
  We define the canonical retraction $f_{Var} : Var \to \Sigma_{Pro}$ by $f_{Var} := RetrRgt \circ RetrList$.
  %% This is a bad hack, but I think it looks nice...
  %% This hack assumes that $\MS{LAM}$ is the wider than of $\MS{RET}$ and $\MS{APP}$
  \begin{alignat*}{1}
    & \MS{SplitBodyStep} := \\
    & \quad \MS{If}~(\LiftTapes{\MS{MatchList}}{\Vector{0;3}}) \\
    & \quad \MS{Then}~\Match~(\LiftBoth{\MS{MatchCom}}{RetrList}{\Vector{3}}) \\
    & \quad \phantom{\MS{Then}}~(\lambda (t : \Option(ACom)).~\MS{match}~y\\
    & \quad \phantom{\MS{Then}~(\lambda}~[~ \phalign{\Some{\MS{LAM}}}{\Some{\MS{RET}}} \Rightarrow \MS{If}~(\LiftBoth{\MS{MatchNat}}{f_{Var}}{\Vector{2}}) \\
    & \quad \phantom{\MS{Then}~(\lambda~[~ \Some{\MS{LAM} \Rightarrow}}~\MS{Then}~ \Return{(\LiftTapes{\MS{AppACom~\MS{RET}}}{\Vector{1;4}})}{\None} \\
    & \quad \phantom{\MS{Then}~(\lambda~[~ \Some{\MS{LAM} \Rightarrow}}~\MS{Else}~~~ \Return{(\LiftTapes{\MS{Reset}}{\Vector{2}})}{\Some\true} \\
    & \quad \phantom{\MS{Then}~(\lambda}~|~ \phalign{\Some{\MS{LAM}}}{\Some{\MS{LAM}}} \Rightarrow \Return{(\LiftBoth{\MS{ConstrS}}{f_{Var}}{\Vector{2}} \Seq \MS{AppACom}~\MS{LAM})}{\None} \\
    & \quad \phantom{\MS{Then}~(\lambda}~|~ \phalign{\Some{\MS{LAM}}}{\Some{\MS{APP}}} \Rightarrow \Return{(\MS{AppACom}~\MS{APP})}{\None} \\
    & \quad \phantom{\MS{Then}~(\lambda}~|~ \phalign{\Some{\MS{LAM}}}{\None} \Rightarrow \Return{(\LiftBoth{\MS{ConstrVar}}{f_{Var}}{\Vector{3}} \Seq \LiftTapes{\MS{AppCom}}{1;3;4})}{\None} \\
    & \quad \phantom{\MS{Then}~(\lambda}~]) \\
    & \quad \MS{Else}~~~ \Return{\Nop}{\Some\false}.
  \end{alignat*}
  % Ok-ish version, without the hack for aligning the "=>"s
  % \begin{alignat*}{1}
  %   & \MS{SplitBodyStep} := \\
  %   & \quad \MS{If}~(\LiftTapes{\MS{MatchList}}{\Vector{0;3}}) \\
  %   & \quad \MS{Then}~\Match~(\LiftBoth{\MS{MatchCom}}{RetrList}{\Vector{3}}) \\
  %   & \quad \phantom{\MS{Then}}~(\lambda (t : \Option(ACom)).~\MS{match}~y\\
  %   & \quad \phantom{\MS{Then}~(\lambda}~[~ \Some{\MS{RET}} \Rightarrow \MS{If}~(\LiftBoth{\MS{MatchNat}}{RetrRgt \circ RetrList}{\Vector{2}}) \\
  %   & \quad \phantom{\MS{Then}~(\lambda~[~ \Some{\MS{RET} \Rightarrow}}~\MS{Then}~ \Return{(\LiftTapes{\MS{AppACom~\MS{RET}}}{\Vector{1;4}})}{\None} \\
  %   & \quad \phantom{\MS{Then}~(\lambda~[~ \Some{\MS{RET} \Rightarrow}}~\MS{Else}~~~ \Return{(\LiftTapes{\MS{Reset}}{\Vector{2}})}{\Some\true} \\
  %   & \quad \phantom{\MS{Then}~(\lambda}~|~ \Some{\MS{LAM}} \Rightarrow \Return{(\LiftBoth{\MS{ConstrS}}{RetrRgt \circ RetrList}{\Vector{2}} \Seq \MS{AppACom}~\MS{LAM})}{\None} \\
  %   & \quad \phantom{\MS{Then}~(\lambda}~|~ \Some{\MS{APP}} \Rightarrow \Return{(\MS{AppACom}~\MS{APP})}{\None} \\
  %   & \quad \phantom{\MS{Then}~(\lambda}~|~ \None \Rightarrow \Return{(\LiftBoth{\MS{ConstrVar}}{RetrRgt \circ RetrList}{\Vector{3}} \Seq \LiftTapes{\MS{AppCom}}{1;3;4})}{\None} \\
  %   & \quad \phantom{\MS{Then}~(\lambda}~]) \\
  %   & \quad \MS{Else}~~~ \Return{\Nop}{\Some\false}.
  % \end{alignat*}
  We furthermore define $\MS{SplitBodyLoop} := \While~\MS{SplitBodyStep}$ and
  \[
    \MS{SplitBody} := \LiftTapes{\MS{ConstrNil}}{\Vector{1}} \Seq \LiftBoth{\MS{ConstrO}}{f_{Var}}{\Vector{2}} \Seq \MS{SplitBodyLoop}.
  \]
\end{definition}

\begin{table}[t]
  % \centering
  \begin{tabular}{l||l|l|l|l}
    Input       & $\MS{MatchList}$ & $\MS{MatchCom}$ & $\MS{MatchNat}$ & $\MS{AppACom}~\MS{RET}$ \\ \hline
    $0:P$       & $0:P'$           &                 &                 &                         \\
    $1:Q$       &                  &                 &                 & $0: Q \app [\MS{RET}]$  \\
    $2:k$       &                  &                 & $0: k'$         &                         \\
    $3:~\dashv$ & $1: \MS{RET}$    & $0:~\dashv$     &                 &                         \\
    $4:~\dashv$ &                  &                 &                 & $1:~\dashv$             \\
  \end{tabular}
  \caption{Execution protocol of $\MS{SplitBodyStep}$ for $P=\MS{RET} \cons P'$ and $k=S~k'$.  The step machine terminates in the partition $\None$,
    thus the loop continues.  Note that tape~$4$ is only used as an internal tape for $\MS{AppACom}$.}
  \label{tab:exec-JumpTarget-RET}
% \end{table}
% \begin{table}
%   \centering
  \begin{tabular}{l||l|l|l|l}
    Input       & $\MS{MatchList}$ & $\MS{MatchCom}$ & $\MS{ConstrVar}$ & $\MS{AppCom}$              \\ \hline
    $0:P$       & $0:P'$           &                 &                  &                            \\
    $1:Q$       &                  &                 &                  & $0: Q \app [\MS{VAR}(n)]$  \\
    $2:k$       &                  &                 &                  &                            \\
    $3:~\dashv$ & $1: \MS{VAR}(n)$ & $0: n$          & $0: \MS{VAR}(n)$ & $1:~\dashv$                \\
    $4:~\dashv$ &                  &                 &                  & $2:~\dashv$                \\
  \end{tabular}
  \caption{Execution protocol of $\MS{SplitBodyStep}$ for $P=\MS{VAR}(n) \cons P'$.  The step machine terminates in the partition $\None$.}
  \label{tab:exec-JumpTarget-VAR}
\end{table}

Execution protocols for the step machine are visualised in Table~\ref{tab:exec-JumpTarget-RET} and Table~\ref{tab:exec-JumpTarget-VAR}.

Note that the step machine resets the tape for $k$ before it breaks out of the loop and returns $\true$.  Thus, no resetting is needed after the loop.
The loop machine halts in $\true$ if and only if $\phi'~k~Q~P$ is not $\None$.  If it is $\Some{(P',Q')}$, then after the execution the first two
tapes contain $P'$ and $Q'$, and the other tapes are right.  If $\phi'~k~Q~P = \None$, the correctness relation only commits that the machine
terminates in the partition $\false$.
\begin{lemma}[Correctness of $\MS{SplitBodyLoop}$][JumpTarget_Loop_Realise]
  $\MS{SplitBodyLoop} \Realise SplitBodyLoopRel$.
  \begin{alignat*}{1}
    & SplitBodyLoopRel := \\
    &\quad \lambda t~(y, t').~\forall P~Q~k.~t[0] \simeq P \rightarrow t[1] \simeq Q \rightarrow t[2] \simeq k \rightarrow isRight~t[3] \rightarrow isRight~t[4] \rightarrow \\
    &\quad\quad \MS{if}~y~\MS{then}~\exists P'~Q'.~ \phi'~k~Q~P = \Some{(Q', P')} \land \\
    &\quad\quad \phantom{\MS{if}~y~\MS{then}}~t'[0] \simeq P' \land t'[1] \simeq Q' \land \forall(i:\Fin_3).~isRigth~t'[2+i] \\
    &\quad\quad \MS{else}~\phi'~k~Q~P=\None.
\end{alignat*}
\end{lemma}

Before the loop, the tapes for $Q$ and $k$ are initialised to $\nil$ and $0$.
\begin{lemma}[Correctness of $\MS{SplitBodyLoop}$][JumpTarget_Realise]
  $\MS{SplitBody} \Realise SplitLoopRel$ with
  \begin{alignat*}{1}
    & SplitBodyRel := \\
    &\quad \lambda t~(y, t').~\forall P.~t[0] \simeq P \rightarrow isRigth~t[1] \rightarrow (\forall(i:\Fin_3).~isRigth~t[2+i]) \rightarrow \\
    &\quad\quad \MS{if}~y~\MS{then}~\exists P'~Q.~ \phi~P = \Some{(Q, P')} \land \\
    &\quad\quad \phantom{\MS{if}~y~\MS{then}}~t'[0] \simeq P' \land t'[1] \simeq Q \land \left(\forall(i:\Fin_3).~isRigth~t'[2+i]\right) \\
    &\quad\quad \MS{else}~\phi~P=\None.
\end{alignat*}
\end{lemma}


\subsection{$\MS{Step}$}
\label{sec:heap-Step}
\setCoqFilename{ProgrammingTuringMachines.TM.LM.StepTM}%


We define a machine $\MS{Step}_\Sigma^{11} : \TM(\Option(\Unit))$ that simulates steps of the heap machine.  The tapes $0$, $1$, and $2$ contain the
control stack $T$, the argument stack $V$, and the heap $H$, respectively.  If the heap machine does a step $(T,V,H) \red (T',V',H')$, $\MS{Step}$
terminates in $\None$ and the resulting tapes contain $T',V',H'$.  Otherwise, if $(T,V,H)$ is a terminating state, $\MS{Step}$ terminates in the
partition $\Some\unit$.  Furthermore, if $\MS{Step}$ terminates in $\None$, there is a successor state $(T',V',H')$.

We implement auxiliary machines for each of the three step rules.  $\MS{Step}$ first destructs $T$.  For example, in the case that
$T=(a, (\MS{APP} \cons P)) \cons T'$, the auxiliary machine for $\MS{APP}$ destructs $V$ further and then calls a machine $\MS{App}$ that realises
list-appending and a machine $\MS{Lenght}$ that computes the length of a list.  After that, it pushes the new closures to $T$ and $V$.  As another
example, if $T=(a, (\MS{VAR}(n) \cons P)) \cons T'$, the auxiliary machine for $\MS{VAR}$ calls $\MS{Lookup}$ and pushes the output closure to the
argument stack and the reset closure to the control stack.  If $\MS{Lookup}$ failed, $\MS{Step}$ immediately terminates in $\Some\unit$, indicating
that the state was an halting state.

The machine $\MS{Step} : \TM_{\Sigma^+}^{11}(\Option(\Unit))$ is defined on an arbitrary alphabet $\Sigma^+$ with retractions
$f_{Stack} : \Sigma_{List(Clos)} \hookrightarrow \Sigma$ and $f_{Heap} : \Sigma_{Heap} \hookrightarrow \Sigma$.  We omit the definition of $\MS{Step}$
here.  Although it is quite complex, no new innovations are required, and the designing of the machine was reasonable easy.  The machine first matches
on the control stack.  If it is empty, it immediately terminates in $\Some\unit$.

\begin{lemma}[Correctness of $\MS{Step}$][Step_Realise]
  \label{lem:Step_Realise}
  $\MS{Step} \Realise StepRel$ with
  \begin{alignat*}{1}
    & StepRel := \\
    &\quad \lambda t~(y, t').~\forall T~V~H.~ t[0] \simeq T \rightarrow t[1] \simeq V \rightarrow t[2] \simeq H \rightarrow \left(\forall (i:\Fin_8).~isRight~t[3+i]\right) \rightarrow \\
    &\quad \MS{if}~y=\None~\MS{then}~\exists T'~V'~H'.~(T,V,H) \red (T',H',V') ~\land~ \\
    &\quad \phantom{\MS{if}~y=\None~\MS{then}}~ t'[0] \simeq T' \land t'[1] \simeq V' \land t'[2] \simeq H' \land \left(\forall (i:\Fin_8).~isRight~t'[3+i]\right) \\
    &\quad \MS{else}~ halt(T,V,H) ~\land~ \\
    &\quad \phantom{\MS{else}}~T=\nil \rightarrow t'[0] \simeq \nil \land t'[1] \simeq V \land t'[2] \simeq H \land \left(\forall (i:\Fin_8).~isRight~t'[3+i]\right)
  \end{alignat*}
\end{lemma}

For each developed machine, we also have running time proofs, but we omitted the running time lemmas for shortness.  We have a function
\[
  stepSteps : \List(HClos) \to \List(HClos) \to Heap \to \Nat
\]
such that $\MS{Step}$ terminates in $stepSteps~T~V~H$ steps when the first three tapes contain $T$, $V$, $H$, and the internal tapes are right.
\begin{lemma}[Runtime of $\MS{Step}$][Step_Terminates]
  \label{lem:Step_Terminates}
  $\MS{Step} \TerminatesIn StepT$ with
  \begin{alignat*}{3}
    & StepT &~:=~& \lambda t~k.~\exists T~V~H.~t[0] \simeq V \land t[1] \simeq T \land t[2] \simeq H \land \left(\forall i:\Fin_8.~isRight~t[3+i]\right) \land \\
    &       &    & \phantom{\lambda t~k.}~stepSteps~T~V~H \leq k
  \end{alignat*}
\end{lemma}


\subsection{$\MS{Loop}$}
\label{sec:heap-Loop}
\setCoqFilename{ProgrammingTuringMachines.TM.LM.HaltingProblem}%

We instantiate $\Sigma := \Sigma_{\List(Clos)} + \Sigma_{Heap}$ and define $\MS{Loop} := \While~\MS{Step}$.  This machine terminates if and only if
the heap state that is encoded on the tape of $\MS{Loop}$ terminates.  Moreover, if the heap state terminates with an empty control stack
$(\nil, V', H')$, the tapes of $\MS{Loop}$ contain $(\nil, V', H')$ after the execution.

\begin{lemma}[Correctness of $\MS{Loop}$][Loop_Realise]
  \label{lem:Loop_Realise}
  $\MS{Loop} \Realise LoopRel$ with
  \begin{alignat*}{1}
    & LoopRel := \\
    &\quad \lambda t~t'.~\forall T~V~H.~ t[0] \simeq T \rightarrow t[1] \simeq V \rightarrow t[2] \simeq H \rightarrow \left(\forall (i:\Fin_8).~isRight~t[3+i]\right) \rightarrow \\
    &\qquad \exists T'~V'~H'.~ \left(T,V,H\right) \terminates^* \left(T',V',H'\right) \land \\ 
    &\qquad T' = nil \rightarrow t'[0] \simeq \nil \rightarrow t'[1] \simeq V' \rightarrow t'[2] \simeq H' \rightarrow \left(\forall (i:\Fin_8).~isRight~t'[3+i]\right)
  \end{alignat*}
\end{lemma}

For the running time function $loopSteps$, we need the $step$ and $halt$ functions from Lemma~\ref{lem:heap-red}.  The running time-function of $\MS{Loop}$ is
defined per recursion over the number $k$ of reduction steps of the heap machine.
\begin{lemma}[Runtime of $\MS{Loop}$][Loop_Terminates]
  \label{lem:Loop_Terminates}
  $\MS{Loop} \TerminatesIn LoopT$ with
  \begin{alignat*}{2}
    &LoopT &~:=~& \lambda t~k.~\exists T~V~H.~\exists T'~V'~H'~k'.~ \left(T,V,H\right) \terminates^{k'} \left(T',V',H'\right) \\
    &      &    & \phantom{\lambda t~k.}~t[0] \simeq T \land t[1] \simeq V \land t[2] \simeq H \land \left(\forall i:\Fin_8.~isRight~t[3+i]\right) \land \\
    &      &    & \phantom{\lambda t~k.}~loopSteps~T~V~H~k' \leq k
  \end{alignat*}
  with
  \small
\begin{lstlisting}[style=semicoqstyle]
$loopSteps~T~V~H~k := ~$
  match$k$
  [$0$ => $stepSteps~T~V~H$
  |$S~k'$ =>
   match$step(T, V, H)$
   [$\Some{(T',V',H')}$ =>
    if $halt(T',V',H')$
    then $1 + stepSteps~T~V~H + stepSteps~T'~V'~H'$
    else $1 + stepSteps~T~V~H + loopSteps~T'~V'~H'~k'$
   |$\None$ => $stepSteps~T~V~H$
   ]
  ].
\end{lstlisting}
\end{lemma}

Lemma~\ref{lem:Loop_Realise} says that if $\MS{Loop}$ with tapes that encode $(T,V,H)$ terminates, then heap machine also terminates.  Dually,
Lemma~\ref{lem:Loop_Terminates} says that if heap machine terminates, so does $\MS{Loop}$ when we encode $(T,V,H)$ on the tapes.

The last step of the reduction of the halting problem of the heap machine to the halting problem of multi-tape Turing machines is to define a function
\[
  initTapes : \left(\List(Clos) \times \List(Clos) \times Heap\right) \to \Tape_\Sigma^{11}
\]
such that the first three tapes of $initTape(T,V,H)$ contain $T$, $V$, or $H$, respectively, and the rest tapes are right.

\begin{definition}[$initTapes$][initTapes]
  \label{def:initTapes}
  ~
  \[
    initTapes(T,V,H) := initValue(T) \cons initValue(V) \cons initValue(H) \cons initRight^8
  \]
  \setCoqFilename{ProgrammingTuringMachines.TM.Code.CodeTM}%
  With $\coqlink[initValue]{initValue}(x) := \MS{midtape}~\nil~\MS{START}~(encode(x) \app [\MS{STOP}])$ and $initRight := \MS{midtape}~\nil~\MS{STOP}~\nil$.
  \setCoqFilename{ProgrammingTuringMachines.TM.LM.HaltingProblem}%
\end{definition}

\begin{theorem}[Halting problem reduction][HaltingProblem]
  \label{lem:HaltingProblem}
  Let $(T,V,H)$ be a heap state.  Then
  \[
    \left(\exists T'~V'~H'.~(T,V,H) \terminates^* (T',V',H') \right) \iff \left(\exists c~k.~\MS{Loop}(initTapes(T,V,H)) \terminates^k c)\right)
  \]
\end{theorem}







%%% Local Variables:
%%% mode: LaTeX
%%% TeX-master: "thesis"
%%% End: