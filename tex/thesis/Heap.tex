\chapter{Simulating a Heap Machine}
\label{chap:heap}

In this chapter, we use all techniques, to implement a Turing machine that simulates an abstract heap machine.  First, we define the semantics of this
abstract machine.  After that, we implement a Turing machine that simulates this abstract machine and prove their correctness and polynomial runtime.

\section{Heap Machine}
\label{sec:heap-def}

\todo{Citation for this machine}

Definition of the heap machine.

\begin{definition}[Types of the abstract machine]
  ~
  \begin{alignat*}{2}
    &Tok  &~::=~& \MS{VAR}(n:\Nat) ~|~ \MS{APP} ~|~ \MS{LAM} ~|~ \MS{RET} \\
    &Pro  &~:=~& \List(Pro) \\
    &Add  &~:=~& \Nat \\
    &Clos &~:=~& Add \times Pro \\
    &Entr &~:=~& \Option(Clos \times Add) \\
    &Heap &~:=~& \List(Entr)
  \end{alignat*}
  The configurations of the abstract machine are triples:
  \[
    \Conf := \List(Clos) \times \List(Clos) \times Heap.
  \]
\end{definition}

\begin{definition}[Semantics of the heap machine]
  We define the relation $\succ$ over configurations, as an inductive predicate:
  \begin{alignat*}{2}
    & \inferrule{\phi~0~\nil~P = \Some{(Q, P')}}{\left((a, (\MS{LAM} \cons P)) \cons T, V, H\right) \succ \left((a, P') \cons_{tr} T, (a,Q) \cons V, H\right) }
    \qquad& \MS{LAM} & \\
    & \inferrule{put~H~\Some{(g,b)} = (c,H')}{\left((a, (\MS{APP} \cons P)) \cons T, g \cons (b, Q) \cons V, H\right) \succ \left((c, Q) \cons (a,P) \cons_{tr} T, V, H'\right)}
    \qquad& \MS{APP} \\
    & \inferrule{H[a;n]=\Some{g}}{\left((a, (\MS{VAR}(n) \cons P)) \cons T, V, H\right) \cons \left((a, P) \cons_{tr} T, g \cons V, H\right)}
    \qquad& \MS{VAR} &
  \end{alignat*}
  with $ put~H~c := (\length{H}, H \app [c]) $. $\phi~P :=~ \phi'~0~\nil~P$ is defined with the helping function
  $\phi' : \Nat \to Pro \to Pro \to \Option(Pro \times Pro)$, which is defined by recursion on $P:Pro$.
  \begin{alignat*}{4}
    &\phi'&~0~   &~Q&~(\MS{RET} \cons P) &&~:=~& \Some{(Q,P')} \\
    &\phi'&~(S~k)&~Q&~(\MS{RET} \cons P) &&~:=~& \phi'~k~(Q \app [\MS{RET}])~P \\
    &\phi'&~k    &~Q&~(\MS{LAM} \cons P) &&~:=~& \phi'~(S~k)~(Q \app [\MS{LAM}])~P \\
    &\phi'&~k    &~Q&~(t        \cons P) &&~:=~& \phi'~k~(Q \app [t])~P \\
    &\phi'&~k    &~Q&~              \nil &&~:=~& \None \\
  \end{alignat*}
  The heap lookup function $H[a,n] : \Option(Clos)$ is defined by recursion over $n$.
  \[
    H[a,n] :=
    \begin{cases}
      \Some{g}  & H[a]=\Some{\Some{(g,b)}} \land n=0 \\
      H[b, n-1] & H[a]=\Some{\Some{(g,b)}} \land n>0 \\
      \None     & H[a]=\None
    \end{cases}
  \]
  where $H[\cdot] : \Option(\Nat)$ is the list lookup function.
\end{definition}


\section{Implementation As Turing Machine}
\label{sec:heap-implementation}

We use the ``programming'' techniques from Chapter~\ref{chap:programming}.  However, for simplicity, we do not use the strict definition of functional
computation.  The machines only save the input-values that the caller machines really need.  For example, we have a machine $\MS{Nth'}$ that saves the
list, but not the index.  The second change is that for functions with optional outputs, like $nth$, we do not extend the alphabet with
$SigOpt~\Sigma_X$.  Instead we encode whether the output is $\Some{\cdot}$ or $\None$ with the partition $\Bool$.  In case that the output is $\None$,
the machine terminates in $\false$ and the output tape stays right.  Note however, that these changes are only made for simplicity, because the
machine we present in this chapter is fairly complex.




\section{Reduction of the Halting Problem}
\label{sec:halting-problem}

We can refine terms of closed lambda expressions to states of the abstract heap machine.  Hence, the halting problemm of lambda terms reduces to the
halting problem of multi-tape Turing machines.

\todo{Kommt hier auch die Reduktion von (closed) Lambda-Termen nach Heap rein?}



%%% Local Variables:
%%% mode: LaTeX
%%% TeX-master: "thesis"
%%% End:
