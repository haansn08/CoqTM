\chapter{Simulating a Heap Machine}
\label{chap:heap}

In this chapter, we use all techniques, to implement a Turing machine that simulates an abstract heap machine.  First, we briefly define the semantics
of this abstract machine.  After that, we implement a Turing machine that simulates this machine and prove its correctness and polynomial runtime
overhead.  We show that this machine reduces the halting problem of the heap machine to the halting problem of multi-tape Turing machines.

\section{Heap Machine}
\label{sec:heap-def}

The abstract machine we present here is a variant of the heap machine in Kunze et~al.~\cite{KunzeEtAl:2018:Formal}.  Idem show that terms of the weak
call-by-value lambda-calculus can be refined to states of said abstract machine and formalise their results in Coq.
\todo{Should I include the compilation function $\gamma$ from lambda-terms to programs?}

Values of this machines are closures.  A closure is a program and a pointer to the environment of bindings.  This environment is implemented as a
linked list of closures.  Programs are lists of tokens, as defined below.
\begin{definition}[Programs, closures, and heaps]
  \label{def:Heap}
  ~
  \begin{alignat*}{2}
    &Var  &~:=~& \Nat \\
    &Tok  &~::=~& \MS{VAR}(n:Var) ~|~ \MS{APP} ~|~ \MS{LAM} ~|~ \MS{RET} \\
    &Pro  &~:=~& \List(Tok) \\
    &Add  &~:=~& \Nat \\
    &Clos &~:=~& Add \times Pro \\
    &Entr &~:=~& \Option(Clos \times Add) \\
    &Heap &~:=~& \List(Entr)
  \end{alignat*}
  We use the letters $n:Var$, $t:Tok$, $P,Q:Pro$, $a,b:Var$, $g:Clos$, $e:Entr$, and $H:Heap$.
\end{definition}

The states of the abstract machine are triples:
\[
  \Conf := \List(Clos) \times \List(Clos) \times Heap.
\]
The first list is the control stack and the second list is the value stack.  We define the small-step operational semantics of this machine as a
binary relation $\succ$ between states of the machine.
\begin{definition}[Semantics of the heap machine]
  {\small
    \begin{alignat*}{4}
      & \left((a, (\MS{LAM} \cons P)) \cons T, V, H\right)                      &~\succ~& \left((a, P') \cons_{tr} T, (a,Q) \cons V, H\right)
      \quad&& \text{if $\phi~P = \Some{(Q, P')}$} & \\
      & \left((a, (\MS{APP} \cons P)) \cons T, g \cons (b, Q) \cons V, H\right) &~\succ~& \left((c, Q) \cons (a,P) \cons_{tr} T, V, H'\right)
      \quad&& \text{if $put~H~\Some{(g,b)} = (c,H')$} \\
      & \left((a, (\MS{VAR}(n) \cons P)) \cons T, V, H\right)                   &~\succ~& \left((a, P) \cons_{tr} T, g \cons V, H\right)
      \quad&& \text{if $H[a,n]=\Some{g}$} &
    \end{alignat*}
  }
  with $put~H~c := (\length{H}, H \app [c]) $.\\
  We define $\phi~P :=~ \phi'~0~\nil~P$ with the auxiliary function $\phi' : \Nat \to Pro \to Pro \to \Option(Pro \times Pro)$, which is defined by
  recursion on $P:Pro$.
  \begin{alignat*}{4}
    &\phi'&~0~   &~Q&~(\MS{RET} \cons P) &&~:=~& \Some{(Q,P')} \\
    &\phi'&~(S~k)&~Q&~(\MS{RET} \cons P) &&~:=~& \phi'~k~(Q \app [\MS{RET}])~P \\
    &\phi'&~k    &~Q&~(\MS{LAM} \cons P) &&~:=~& \phi'~(S~k)~(Q \app [\MS{LAM}])~P \\
    &\phi'&~k    &~Q&~(t        \cons P) &&~:=~& \phi'~k~(Q \app [t])~P \\
    &\phi'&~k    &~Q&~              \nil &&~:=~& \None \\
  \end{alignat*}
  The heap lookup function $H[a,n] : \Option(Clos)$ is defined by recursion on $n$.
  \[
    lookup~H~a~n :=
    \begin{cases}
      \Some{g} & H[a]=\Some{\Some{(g,b)}} \land n=0 \\
      lookup~H~b~(n-1) & H[a]=\Some{\Some{(g,b)}} \land n>0 \\
      \None & \text{else}
    \end{cases}
  \]
  where $H[\cdot] : \Option(Entr)$ is the list lookup function.  We write $H[a,n] := lookup~H~n$. \\
  The tail-recursion optimised cons operation $\cons_{tr}$ is defined as follows:
  \begin{alignat*}{3}
    & (a, \nil) &~\cons_{tr}~& H &~:=~& H \\
    & (a, P   ) &~\cons_{tr}~& H &~:=~& (a, P) \cons H \qquad\text{if $P \neq \nil$}
  \end{alignat*}
\end{definition}

We write $\succ^k$ for the relational power (cf.\ Definition~\ref{def:pow}) and $\succ^*$ for the reflexive transitive closure of $\succ$ (cf.\
Definition~\ref{def:Kleene}).

The function $\phi$ matches $\MS{LAM}$ with $\MS{RET}$.  It splits the program into two parts, the first part is the ``body'' of a linearised lambda
expression and the second part the rest of the program, for example:%
{\small
  \[
    \phi[\MS{VAR}(0); \MS{LAM}; \MS{APP}; \MS{RET}; \MS{RET}; \MS{VAR}(1)] = \Some{([\MS{VAR}(0); \MS{LAM}; \MS{APP}; \MS{RET}], [\MS{VAR}(1)])}.
  \]
}%
The first $\MS{RET}$ matches to the $\MS{LAM}$ in the program above, so it is part of the first partition.  When the heap machine has a $\MS{LAM}$
token as first control token, it finds the rest of the program using $\phi$ and pushes the body on the control stack.

When the first control token is $\MS{App}$, the machine fetches two closures from the value stack.  The first closure, $g$, corresponds to the
parameter of the application.  The second closure, $(b,Q)$, corresponds to the called function.  The machine pushes $g$ on the heap.  After that it
continues executing $Q$, by adding a new closure $(c,Q)$ to the control stack, where $c$ is the updated head of the heap.

If the first control token is $\MS{VAR}(n)$, the machine looks up the $n$th entry in the environment on the heap at the address $a$, and pushes this
closure to the value stack.


\section{Implementation As Turing Machine}
\label{sec:heap-implementation}

First, we notice that all types in Definition~\ref{def:Heap}, except program Tokens ($Tok$), are encodable over canonical alphabets, according to
Definition~\ref{def:basic-encodings}.
\begin{definition}[Encoding of heaps]
  \label{def:Heap-encode}
  The type of tokens ($Tok$) is isomorphic to the sum of $ATok := \MS{APP} ~|~ \MS{LAM} ~|~ \MS{RET}$ and $\Nat$.  $ATok$ is encodable on itself
  (since it is finite), and $Var = \Nat$ is encodable on $\Sigma_{Var} := \Sigma_\Nat$.  Then $Tok$ is encodable on
  $\Sigma_{Tok} := \Sigma_{ATok+\Nat}$.  All other types are encodable according to Definition~\ref{def:basic-encodings}.
  For completeness, the alphabets are:\\
  $\Sigma_{Pro} := \Sigma_{\List(Tok)}$, $\Sigma_{Add} := \Sigma_\Nat$, $\Sigma_{Clos}:=\Sigma_{Add \times Pro}$,
  $\Sigma_{Entr} := \Sigma_{\Option(Clos \times Add)}$, and $\Sigma_{Heap} := \Sigma_{\List(Entr)}$.
\end{definition}

For the functions $\phi$ and $lookup$, we define machines that compute these function in a lesser strict sense than
Definition~\ref{def:Computes2_Rel}.  The Turing machines only save the input-values that a caller machine needs again.  For example, we have a machine
$\MS{Nth'}$ that saves the list, but not the index.  The second change is that for functions with optional outputs, like $nth$, we do not extend the
alphabet with $\Sigma_{\Option(X)}$.  Instead, we encode whether the output is $\Some{\cdot}$ or $\None$ with the partition $\Bool$.  In case that the
output is $\None$, the machine terminates in $\false$ and the output tape stays right.

We define a machine $\MS{Step} : \TM(\Option(\Unit))$ (we specify the alphabet and the tape number later) that simulates steps of the heap machine.
The tapes $0$, $1$, and $2$ encode the control stack, the value stack, and the heap, respectively.  If the heap machine does a step $s \succ s'$, then
if the tapes contain $s$, $\MS{Step}$ terminates in $\None$ and the resulting tapes contain $s'$.  Otherwise, if $s$ is a terminating state,
$\MS{Step}$ terminates in the partition $\Some\unit$.  For that, we implement auxiliary machines for each of the three step rules.  Lastly, we define
$\MS{Loop} := \While~\MS{Step}$.  This machine terminates if and only if the heap state that is encoded on the tape of $\MS{Loop}$ terminates.  We
include execution protocols but omit definitions of machines and correctness relations, as they can be derived from the execution protocols.  This
means that we prove that the halting problem of the heap machine reduces to the halting problem of $n$-tape Turing machines.


\subsection{$\MS{Lookup}$}
\label{sec:Lookup}

The machine $\MS{Lookup} : \TM_\Sigma^5(\Bool)$ realises the heap $lookup$ function.  Note, that the heap is just a linked list of closures, where the
pointers in this linked lists are indices.  It uses that machine $\MS{Nth}' : \TM_\Sigma^4(\Bool)$, which is $\MS{Nth}$ from Section~\ref{sec:Nth}
with the two changes mentioned above.

First we consider the alphabet of $\MS{Lookup}$.  We can encode closures on the heap alphabet $\Sigma_{Heap}$, but when $\MS{Step}$ calls $H[a,n]$,
$a$ is the current heap address from the first closure on the control stack, and $n$ is the variable the first variable token of the program of that
closure.  Thus, $a$ and $n$ are encoded on a different closure alphabet, than the closures on the heap.  We define $\MS{Lookup}$ on an alphabet
$\Sigma$, with retractions $f_{Heap} : \Sigma_{Heap} \hookrightarrow \Sigma$ and $f_{Clos} : \Sigma_{Clos} \hookrightarrow \Sigma$.  The second
retraction correspond to the encoding of $a$ and $n$.  The retraction
\[ f_{Clos'} : \Sigma_{Clos} \hookrightarrow \Sigma := RetrFst \circ RetrOpt \circ RetrList \circ f_{Heap} \]%
is the canonical retraction of closures from the heap.

There are three relevant retractions $\Sigma_\Nat \hookrightarrow \Sigma$, depending on whether the number is an heap pointer in the heap alphabet,
or a variable or heap pointer from the stack alphabet.
\begin{enumerate}
\item Address of the pointer of an heap on the stack alphabet:
  \[
    f_{add} := RetrFst \circ f_{Clos'}
  \]
\item Variable in a token of a closure on the stack alphabet:
  \[
    f_{var} := RetrRgt \circ RetrList \circ RetrSnd \circ f_{Clos'}
  \]
  Note that $RetrRgt : \Sigma_\Nat \hookrightarrow \Sigma_{Tok}$.
\item Pointer to the next heap entry:
  \[
    f_{next} := RetrSnd \circ RetrOpt \circ RetrList \circ f_{Heap}
  \]
\end{enumerate}

Since the function $lookup$ is tail-recursive, we define the step machine with the type $LookupStep : \TM_\Sigma^5(\Option(\Bool))$.  It first calls
the list lookup machine ($\MS{Nth'}$).  If it failed, $\MS{Lookup}$ immediately terminates $\false$.  Else, it makes a case-distinction on the heap
entry $e : \Option(Clos \times Add)$.  In case it is $\None$, $\MS{Lookup}$ terminates in $\false$.  Else it destructs the closure $g = (g,b)$ and
makes a case-distinction over $n$.  If $n=0$, $\MS{LookupStep}$ resets the tapes for $b$ and $n$, translates $g$ from $f_{Clos'}$ to $f_{Clos}$ and
returns $\true$.  Else it translates $b$ from $f_{next}$ to $f_{add}$, moves it to the tape that contained $a$, and repeats the loop.  We visualise
the execution in the latter two cases in Tables~\ref{tab:exec-LookupStep-S} and~\ref{tab:exec-LookupStep-0}.

\begin{table}[t]
  \centering
  \scriptsize
  \begin{tabular}{l||l|l|l|l|l|l|l|l}
    Input      & $\MS{Nth'}$      & $\MS{MatchOpt}$       & $\MS{MatchPair}$     & $\MS{MatchNat}$ & $\MS{CopyValue}$  & $\MS{Translate}$ & $\MS{Reset}$ & $\MS{Reset}$ \\ \hline
    $0:H$      & $0:H$            &                       &                      &                 &                   &                  &              &              \\
    $1:a$      & $1:\dashv$       &                       &                      &                 & $1:b$             & $0:b$            &              &              \\
    $2:n$      &                  &                       &                      & $0:n'$          &                   &                  &              &              \\
    $3:\dashv$ &                  &                       & $1: g$               &                 &                   &                  &              & $0:\dashv$   \\
    $4:\dashv$ & $2:\Some{(g,b)}$ & $0:(g,b)$             & $0:b$                &                 & $0:b$             &                  & $0:\dashv$   &              \\
  \end{tabular}
  \caption{Execution protocol of $\MS{LookupStep}$ in case $H[a] = \Some{\Some{(g,b)}}$ and $n=S~n'$.  It terminates in the partition $\None$.  The
    translation is from $f_{next}$ to $f_{add}$.}
  \label{tab:exec-LookupStep-S}
\end{table}
\begin{table}[t]
  \centering
  \scriptsize
  \begin{tabular}{l||l|l|l|l|l|l|l}
    Input      & $\MS{Nth'}$      & $\MS{MatchOpt}$       & $\MS{MatchPair}$     & $\MS{MatchNat}$ & $\MS{Reset}$ & $\MS{Reset}$ & $\MS{Translate}$ \\ \hline
    $0:H$      & $0:H$            &                       &                      &                 &              &              &                  \\
    $1:a$      & $1:\dashv$       &                       &                      &                 &              &              &                  \\
    $2:n$      &                  &                       &                      & $0:n'$          &              & $0:\dashv$   &                  \\
    $3:\dashv$ &                  &                       & $1: g$               &                 &              &              & $0:g$            \\
    $4:\dashv$ & $2:\Some{(g,b)}$ & $0:(g,b)$             & $0:b$                &                 & $0:\dashv$   &              &                  \\
  \end{tabular}
  \caption{Execution protocol of $\MS{LookupStep}$ in case $H[a] = \Some{\Some{(g,b)}}$ and $n=0'$.  It terminates in the partition $\Some\true$.  The
    translation is from $f_{Clos'}$ to $f_{Clos}$.}
  \label{tab:exec-LookupStep-0}
\end{table}

\begin{definition}[$\MS{LookupStep}$]
  \label{def:Lookup_Step}
  ~
  \begin{align*}
    & \MS{LookupStep} := \\
    & \quad \MS{If}~(\MS{Nth'}~f_{heap}~f_{add}) \\
    & \quad \MS{Then}~\MS{If}~(\LiftBoth{\MS{MatchOption}}{RetrList ~\circ~ f_{heap}}{\Vector{4}}) \\
    & \quad \phantom{\MS{Then}}~\MS{Then}~\LiftBoth{\MS{MatchPair}}{RetrOpt \circ RetrList ~\circ~ f_{heap}}{\Vector{4;3}} \Seq \\
    & \quad \phantom{\MS{Then}~\MS{Then}}~\MS{If}~(\LiftBoth{\MS{MatchNat}}{f_{var}}{\Vector{2}}) \\
    & \quad \phantom{\MS{Then}~\MS{Then}~\MS{Then}}~\MS{Then}~ \MS{Return}~\bigl(
      \LiftTapes{\MS{CopyValue}}{\Vector{4;1}} \Seq
      \LiftTapes{\MS{Translate}~f_{next}~f_{ad}}{\Vector{1}} \Seq \\
    & \quad \phantom{\MS{Then}~\MS{Then}~\MS{Then}~\MS{Then}~\MS{Return}\bigl(}
      \LiftTapes{\MS{Reset}}{\Vector{4}} \Seq
      \LiftTapes{\MS{Reset}}{\Vector{3}}
      \bigr)~{\None} \\
    & \quad \phantom{\MS{Then}~\MS{Then}~\MS{Then}}~\MS{Else}~~~ \MS{Return}~\bigl(
      \LiftTapes{\MS{Reset}}{\Vector{4}} \Seq
      \LiftTapes{\MS{Reset}}{\Vector{2}}
      \bigr)~{\Some\true} \\
    & \quad \phantom{\MS{Then}}~\MS{Else}~~~ \Return{\Nop}{\Some\false} \\
    & \quad \MS{Else}~~~ \Return{\Nop}{\Some\false}
  \end{align*}
\end{definition}

The step machine terminates in $\Some\false$ in two cases.  Either the list lookup failed, or it returned the empty heap entry $H[a]=\Some\None$.  In
the successfully termination case (cf.\ Table~\ref{tab:exec-LookupStep-0}), $\MS{LookupStep}$ resets the afterwards unneeded tapes, especially $n$, of
which the value may have been overwritten.

Because no initialisation or clean-up is needed before or after the loop, we can define $\MS{Lookup} := \While~\MS{LookupStep}$.  The correctness
relation of $\MS{Lookup}$ says that if it terminated in the partition $\true$, we have $H[a,n] = \Some{g}$ for some closure $g$, and tape~$3$ contains
$g$ w.r.t.\ the retraction $f_{Clos}$.  Furthermore, the input tape~$0$ still contains $H$ and all other tapes are reseted.  In case $\MS{Lookup}$
terminated in $\false$, the postcondition only asserts that $H[a,n]=\None$.  Note that there two possible reasons for that: Either $H[a]=\None$, or
$H[a]=\Some\None$.  We do not distinct these two failure cases, because we then know that the state of the heap machine is a terminating state.  In
this case, $\MS{Step}$ only needs to terminate in $\Some\unit$.

\begin{lemma}[Correctness of $\MS{Lookup}$]
  \label{lem:Lookup_Realise}
  $\MS{Lookup} \Realise LookupRel$
  with
  \small
  \begin{alignat*}{1}
    & LookupRel := \\
    &\quad \lambda t~(y, t').~\forall H~a~n.~ t[0] \simeq H \rightarrow t[1] \simeq_{f_{add}} a \rightarrow t[2] \simeq_{f_{var}} n \rightarrow isRight~t[3] \rightarrow isRight~t[4] \rightarrow \\
    &\quad\quad \MS{if}~y~\MS{then}~\bigl(\exists g.~H[a,n] = \Some{g} \land \\
    &\quad\quad \phantom{\MS{if}~y~\MS{then}~\bigl(\exists g.~} t'[0] \simeq H \land isRight~t'[1] \land isRight~t'[2] \land t'[3] \simeq_{f_{Clos}} g \land isRight~t[4]\bigr) \\
    &\quad\quad \MS{else}~ H[a,n] = \None.
\end{alignat*}
\end{lemma}

\todo{How to present runtime here?}




%%% Local Variables:
%%% mode: LaTeX
%%% TeX-master: "thesis"
%%% End:
