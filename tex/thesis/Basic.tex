\chapter{Primitive Machines}
\label{chap:basic}

In this chapter, we define several classes of \textit{primitive} machines.\footnote{Asperti and Ricciotti~\cite{asperti2015} call these machines
  \textit{basic} machines.}  These machines are defined on an arbitrary alphabet $\Sigma$, but they have at most one tape, and they terminate after at
most one transition.  Primitive machines are the only machines for that we give transition functions $\delta$ explicitly.  They are also the only
machines, for that we fully specify the type of machine states $Q$.  In the following two chapters, we will show how to combine these machines and
build (and verify) complex machines without mentioning $Q$ or $\delta$ again.


\section{$\MS{Null}$}
\label{sec:Null}

\setCoqFilename{ProgrammingTuringMachines.TM.Basic.Null}


$\MS{Null}$ is a very simple class of machines.  The machine $\MS{Null}_\Sigma$ is parametrised over the alphabet $\Sigma$.  Because $\Sigma$ is
always clear from the context, we leave the index out.  We fix an alphabet $\Sigma$.  $\MS{Null}$ has zero tapes and terminates immediately, i.e.\
after $0$ steps.

\begin{definition}[$\MS{Null}$][Null]
  \label{def:Null}
  $\MS{Null} : \TM_\Sigma^{\mkern+1mu 0}$ is defined as follows:
  \begin{align*}
    Q          &:= \Unit \\
    start      &:= \unit \\
    \delta ~\_ &:= (\unit, \nil) \\
    halt   ~\_ &:= \true
  \end{align*}
\end{definition}
Note that if we have an unlabelled machine $M:\TM_\Sigma^n$, we implicitly label their states over $\Unit$ with the labelling function
$lab_M(q):=\unit$.

% The correctness relation says exactly that the tapes don't change:
The correctness relation is the universal relation, because empty vectors do not have information.  However, the correctness lemma also states that
the machine terminates in $0$ steps.
\begin{lemma}[Correctness of $\MS{Null}$][Null_Sem]
  \label{lem:Null_Sem} $\MS{Null} \RealiseIn0 NullRel$ with $NullRel := \lambda~t~t'.~ \True$.
\end{lemma}
\begin{proof}
  By execution.  The machine terminates after zero steps in the state $\unit$.
\end{proof}

\section{$\MS{DoAct}$}
\label{sec:DoAct}

\setCoqFilename{ProgrammingTuringMachines.TM.Basic.Mono}

Machines of the next class, $\MS{DoAct}~a : \TM_\Sigma^1(L)$, do only one action $a:\Act$ (i.e.\ they optionally write a symbol and move the head of
the tape) and terminate.
\begin{definition}[$\MS{DoAct}~a$][DoAct]
  \label{def:DoAct}
  Let $a:\Act_\Sigma$.  Then $\MS{DoAct}~a : \TM_\Sigma^1$ is defined as follows:
  \begin{align*}
    Q          &:= \Bool \\
    start      &:= \false \\
    \delta ~\_ &:= (\true, \Vector{a}) \\
    halt   ~ b &:= b
  \end{align*}
\end{definition}
The semantics of $\MS{DoAct}$ is easily expressed using the function $\MS{doAct}$ (see Definition~\ref{def:step}).  The machine terminates after one
transition:
\begin{lemma}[Correctness of $\MS{DoAct}$][DoAct_Sem]
  \label{lem:DoAct_Sem} $\MS{DoAct}~a \RealiseIn1 DoActRel~a$ with
  \[
    DoActRel~a := \lambda t~t'.~ t'[0] = \MS{doAct}~t[0]~a.
  \]
\end{lemma}
% \begin{proof}
%   By execution.
% \end{proof}
We define some abbreviations:
\begin{definition}[Machine classes derived from $\MS{DoAct}$][DoAct_Derived]
 \label{def:DoAct-derived} 
 \begin{align*}
   \MS{Move}       ~d &:= \MS{DoAct} (\None, d) \\
   \MS{Write}    ~s   &:= \MS{DoAct} (\Some{s}, N) \\
   \MS{WriteMove}~s~d &:= \MS{DoAct} (\Some{s}, d)
 \end{align*}
\end{definition}


\section{$\MS{Read}$}
\label{sec:basic_machines-Read}

$\MS{Read} : \TM_\Sigma^1(\Option(\Sigma))$ is an interesting class of labelled one-tape machines.  The machines of this class have one terminating
state for each symbol of the alphabet $\Sigma$.  They read the current symbol from the tape and terminate in the state that corresponds to that
symbol.  For the case that there is no current symbol, they also have a distinct terminating state.  The labelling function maps the terminating state
for the symbol $s$ to the label $\Some{s}$.

\begin{definition}[$\MS{Read}$][ReadChar]
  The machine $\MS{Read} : \TM_\Sigma^1(\Option(\Sigma))$ is defined as follows:
  \begin{align*}
    Q          &:= \Bool+\Sigma \\
    start      &:= \inl \false \\
    \delta (\_, s) &:=
                     \begin{cases}
                       (\inl \true, \Vector{(\None, N)}) & s[0] = \None \\
                       (\inr c, \Vector{(\None, N)})     & s[0] = \Some c
                     \end{cases} \\
    halt   ~ (\inl  b) &:= b \\
    halt   ~ (\inr \_) &:= \true \\
    lab    ~ (\inl \_) &:= \None \\
    lab    ~ (\inr  s) &:= \Some s
  \end{align*}
\end{definition}

The correctness lemma of $\MS{Read}$ states that the machine terminates after one step, leaves the tape unchanged, and that the label of the
terminating state corresponds to the current symbol on the tape.

\begin{lemma}[Correctness of $\MS{Read}$][ReadChar_Sem]
  \label{lem:Read_Sem} $\MS{Read} \RealiseIn{1} ReadRel$ with
  \[
    ReadRel := \lambda~t~(l,t').~ l = \MS{current}~t[0] \land t' = t
  \]
\end{lemma}
\begin{proof}
  Case distinction over $\MS{current}~t[0]$.  Both cases follow by executing the machine one step.
\end{proof}


%%% Local Variables:
%%% TeX-master: "thesis"
%%% End: