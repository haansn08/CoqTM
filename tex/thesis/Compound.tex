\chapter{Compound Machines}
\label{cha:compound}

In this chapter, we build complex machines, using the combinators developed in Chapter~\ref{chap:combining}, as well as the tapes-lift of
Chapter~\ref{chap:lifting}.  However, we do not use the alphabet lift in this chapter.  We also prove correctness and runtime of these machines.

The general approach, when we prove correctness of a machine $M$, i.e.\ if we want to show $M \vDash R$, is to first define a relation $R'$ that $M$
realises.  For that, we use the correctness lemmas of the combinators, lifts, and all concrete machines ``involved''.  The main part of the proof is
applying Lemma~\ref{lem:Realise_monotone} and showing $R' \subseteq R$.  Note that the first part of the proof, i.e.\ generating the relation $R'$
such that $M \vDash R'$, is mechanical.  Indeed, we can automatise this step in Coq, see more in Chapter~\ref{chap:implementation}.

When $M$ always terminates in constant time $k$, we show $M \vDash^k R$ instead.  Using the monotonicity Lemma~\ref{lem:RealiseIn_monotone}, we can
prove correctness and (constant) runtime at once.

For (non-constant) runtime, we use the dual approach.  However, we have an exception for $\While$, where we have to apply
Lemma~\ref{lem:While_TerminatesIn} directly instead of using the anti-monotonicity Lemma~\ref{lem:TerminatesIn_monotone}.

\section{$\Nop$}
\label{sec:Nop}

Using the alphabet-lift (Definition~\ref{def:LiftAlphabet}) and $\MS{Null}$ (Definition~\ref{def:Null}), it is easy to define an $n$-tape machine
$\Nop : \TM_\Sigma^n$ for every alphabet $\Sigma$ and number of tapes $n$:
\begin{definition}[$\Nop$]
  $Nop := \LiftTapes{\MS{Null}}{\nil}.$
\end{definition}
Note that because $\MS{Null}$ is a 0-tape machine, and $\Nop$ is supposed to be an $n$-tape machine, the index-vector must be the vector
$\nil : \Fin_n^0$.

\begin{lemma}[Correctness of $\Nop$]
  \label{lem:Nop_Sem}
  $\Nop \vDash^0 NopRel$ with $NopRel := \lambda t~t'.~t'=t$.
\end{lemma}
\begin{proof}
  By applying monotonicity of $\vDash^\cdot$ (Lemma~\ref{lem:RealiseIn_monotone}) and the correctness of $\MS{Null}$ (Lemma~\ref{lem:Null_Sem}), it
  remains to show that $\LiftTapes{NullRel}{\nil} \subseteq NopRel$.  To show the equality $t'=t$ we show $t'[i]=t[i]$ for all $i:\Fin_n$.  This
  follows with the equality part of the relation $\LiftTapes{NullRel}{\nil}$, because $i \notin \nil$.
\end{proof}

Note that the correctness relation of $\Nop$ can also be expressed using the identity relation $Id$:
\[
  \MS{NopRel} \equiv Id.
\]
However, we have the convention to define relations of concrete machine (classes) in $\lambda$-notation, i.e.\ not using relational operators.  Also
note that the tape $t'$ is per convention always on the left side of the equality.  This convention makes rewriting of tapes uniform; therefore,
rewriting of tapes can be automatised in Coq.

\section{$\MS{WriteString}$}
\label{sec:WriteString}

The machine $\MS{WriteString}~d~str$ writes a fixed string $str:\List(\Sigma)$ in the direction $d$.  It is defined by recursion over the string:
\begin{definition}[$\MS{WriteString}$]
  \begin{alignat*}{3}
    &\MS{WriteString}~d~&&(\nil)         &~:=~& \Nop \\
    &\MS{WriteString}~d~&&(s \cons str') &~:=~& \MS{WriteMove}~s~d \Seq \MS{WriteString}~d~str'.
  \end{alignat*}
\end{definition}

Note that this is the our only machine we define per recursion.  However, the way we proof correctness in constant time (depending on the length of
$str$) is still the same.

The relation that $\MS{WriteString}$ ``automatically'' realises in $3 \cdot \length{str}$ steps is the following:
\begin{lemma}
  $\MS{WriteString}~d~str \vDash^{3 \cdot \length{str}} R'~d~str$ with
  \begin{alignat*}{3}
    &\MS{R'}~d~&&(\nil        ) &~:=~& NopRel \\
    &\MS{R'}~d~&&(s \cons str') &~:=~& DoActRel(\Some{s}, d) \circ \MS{R'}~d~str'.
  \end{alignat*}
\end{lemma}
\begin{proof}
  By induction on $str:\List(\Sigma)$, using the monotonicity Lemma~\ref{lem:RealiseIn_monotone}, the correctness of $Nop$ (Lemma~\ref{lem:Nop_Sem}),
  the correctness of $\MS{WriteMove}$ (which is defined by $\MS{DoAct}$; Lemma~\ref{lem:DoAct_Sem}), and correctness of sequential composition for
  constant time (Lemma~\ref{lem:Seq_RealiseIn}).
\end{proof}
\todo{I don't have corollaries yet about seq, and if, and $\vDash^\cdot$.}

We define the actual relation of $\MS{WriteString}$ in terms of a recursive function on tapes:
\begin{lemma}[Correctness of $\MS{WriteString}$]
  \label{lem:WriteString_Sem}
  Let $d:\Move$ and $str:\List(\Sigma)$.
  \[ \MS{WriteString}~d~str \vDash^{4\cdot\length{str}} WriteStringRel~d~str \]
  with
  $WriteStringRel~d~str := \lambda t~t'.~t' = WriteStringFun~d~t~str'$ and
  \begin{alignat*}{3}
    &WriteStringFun~d~t~&&(\nil        ) &~:=~& t \\
    &WriteStringFun~d~t~&&(s \cons str') &~:=~& WriteStringFun~d~\bigl(doAct~t~(\Some{s}, d)\bigr)~str'
  \end{alignat*}
\end{lemma}
\begin{proof}
  We apply the monotonicity Lemma~\ref{lem:RealiseIn_monotone} and have to show\\
  $R'~d~str \subseteq WriteStringRel~d~str$.  We show this by induction on $str$.
\end{proof}



%%% Local Variables:
%%% TeX-master: "thesis"
%%% End: